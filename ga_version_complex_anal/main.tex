\documentclass[11pt, a4paper]{article}

% Preamble: Loading necessary packages
\usepackage[margin=1in]{geometry} % For setting page margins
\usepackage{amsmath}               % For advanced math environments
\usepackage{amssymb}               % For extra math symbols
\usepackage{graphicx}              % To include images
\usepackage{hyperref}              % For clickable links and a modern feel
\usepackage{array}                 % For better table columns

% Document Information
\title{Evaluating a Complex Limit with Geometric Algebra: \\ A Comparative Approach}
\author{Gemini}
\date{\today}

% Start of the document
\begin{document}

\maketitle

\begin{abstract}
This document explores the evaluation of a fundamental limit in complex analysis, $\lim_{z \to i} z$, from two perspectives. First, we use the standard method of complex variables. Second, we employ the framework of Geometric Algebra (GA) to provide a more geometrically intuitive and generalizable solution. The goal is to demonstrate the direct translation between the two notations and highlight the conceptual advantages of the GA approach.
\end{abstract}

\section{The Problem}
We wish to evaluate the limit of the identity function $f(z)=z$ as $z$ approaches the imaginary unit $i$. While trivial in standard analysis, this problem serves as an excellent illustration of foundational concepts.
\begin{itemize}
    \item \textbf{In Complex Notation:} We evaluate $\lim_{z \to i} z$.
    \item \textbf{In Geometric Algebra Notation:} We evaluate $\lim_{Z \to I} Z$.
\end{itemize}

\section{The Geometric Algebra Framework}
In the geometric algebra of a 2D plane ($G_2$), a complex number $z = x + iy$ is represented as a \textbf{multivector} $Z = x + yI$.
\begin{itemize}
    \item $x$ is a scalar (grade-0 element).
    \item $I$ is the unit \textbf{pseudoscalar} of the plane, representing a directed unit area. It is the geometric product of the two orthonormal basis vectors ($I = \mathbf{e}_1\mathbf{e}_2$).
    \item A key property, derived from the geometric product, is that $I^2 = -1$. Therefore, $I$ is the direct geometric equivalent of the imaginary unit $i$.
\end{itemize}

\section{The "All Paths at Once" Approach}
To prove a limit exists in the complex plane, one must show that the same value is approached regardless of the path. We can parameterize all straight-line paths simultaneously using a polar displacement from the limit point.

\subsection{Standard Complex Analysis}
We represent a point $z$ approaching $i$ as:
$$ z = i + re^{i\theta} $$
Here, $r \in \mathbb{R}^+$ is the radial distance from $i$, and $\theta$ is the angle of approach. The limit is found by letting $r \to 0$.

\subsection{Geometric Algebra}
Similarly, we represent a point $Z$ approaching $I$ as:
$$ Z = I + \epsilon e^{I\theta} $$
Here, $\epsilon \in \mathbb{R}^+$ is the magnitude of the displacement, and $u = e^{I\theta} = \cos\theta + I\sin\theta$ is a unit multivector that defines the direction. In GA, $e^{I\theta}$ is known as a \textbf{rotor}, as it generates rotations. The limit is found by letting $\epsilon \to 0$.

\section{Side-by-Side Evaluation}
The following table provides a direct translation of the evaluation process.

\bigskip % Add some vertical space before the table

\noindent % Prevent indentation
\begin{tabular}{| >{\raggedright\arraybackslash}p{0.48\linewidth} | >{\raggedright\arraybackslash}p{0.48\linewidth} |}
    \hline
    \textbf{Complex Notation} & \textbf{Geometric Algebra Notation} \\
    \hline
    \multicolumn{2}{|c|}{\textbf{1. State the Problem}} \\
    \hline
    $\lim\limits_{z \to i} z$ & $\lim\limits_{Z \to I} Z$ \\
    \hline
    \multicolumn{2}{|c|}{\textbf{2. Define a General Path of Approach}} \\
    \hline
    Let $z = i + re^{i\theta}$. The limit is taken as the distance $r \to 0$. The term $e^{i\theta}$ represents the direction. & Let $Z = I + \epsilon e^{I\theta}$. The limit is taken as the magnitude $\epsilon \to 0$. The rotor $e^{I\theta}$ represents the direction. \\
    \hline
    \multicolumn{2}{|c|}{\textbf{3. Apply the Function}} \\
    \hline
    The function is $f(z)=z$. Substituting gives: $f(i + re^{i\theta}) = i + re^{i\theta}$. & The function is $f(Z)=Z$. Substituting gives: $f(I + \epsilon e^{I\theta}) = I + \epsilon e^{I\theta}$. \\
    \hline
    \multicolumn{2}{|c|}{\textbf{4. Evaluate the Limit}} \\
    \hline
    $\lim\limits_{r \to 0} (i + re^{i\theta})$
    \newline The magnitude of the displacement term is $|re^{i\theta}| = r$, which vanishes as $r \to 0$.
    \newline $\lim\limits_{r \to 0} (i + re^{i\theta}) = i + 0 = i$. & 
    $\lim\limits_{\epsilon \to 0} (I + \epsilon e^{I\theta})$
    \newline The magnitude of the displacement term is $|\epsilon e^{I\theta}| = \epsilon$, which vanishes as $\epsilon \to 0$.
    \newline $\lim\limits_{\epsilon \to 0} (I + \epsilon e^{I\theta}) = I + 0 = I$. \\
    \hline
    \multicolumn{2}{|c|}{\textbf{5. Conclusion}} \\
    \hline
    The result is $\mathbf{i}$. It is independent of the angle of approach $\theta$, confirming the limit. & The result is $\mathbf{I}$. It is independent of the directional rotor $e^{I\theta}$, confirming the limit. \\
    \hline
\end{tabular}

\bigskip

\section{Conceptual Discussion: Why GA Matters}
For this 2D problem, the notations are functionally identical. The expression $re^{i\theta}$ is a direct analog of $\epsilon e^{I\theta}$. However, the GA framework provides a deeper geometric insight and is far more general.

\begin{itemize}
    \item \textbf{Geometric Foundation:} In complex analysis, $i$ is defined axiomatically by $i^2 = -1$. In geometric algebra, the equivalent property $I^2 = -1$ is not an axiom but a \textit{consequence} of the geometric definition of the algebra. $I$ is a concrete geometric entity (an oriented plane segment), not an abstract "imaginary" number.

    \item \textbf{Generalizability:} The true power of GA becomes apparent in higher dimensions. Complex numbers are limited to 2D transformations. In 3D, a rotation occurs in a specific \textit{plane}, which is represented by a bivector $B$. A rotation in any arbitrary 3D plane can be described by a rotor $R = e^{B\theta}$. This unified and powerful concept of rotors for describing rotations in any dimension has no simple parallel in standard complex number theory.
\end{itemize}

\section{Conclusion}
Evaluating $\lim_{z \to i} z$ using both methods yields the same result, $i$ (or its GA equivalent, $I$). The parallel structure of the derivation shows how complex analysis can be viewed as a specialized subset of the more general geometric algebra of the plane. While complex analysis is a perfectly tailored tool for 2D problems, geometric algebra provides a more fundamental and scalable framework that keeps the underlying geometry at the forefront.

\end{document}
%%% Local Variables:
%%% mode: latex
%%% TeX-master: t
%%% End:
