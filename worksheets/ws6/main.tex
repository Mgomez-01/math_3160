% --------------------------------------------------------------------
% LaTeX Template for Math Worksheets
% --------------------------------------------------------------------
\documentclass{article}

% --- PACKAGE IMPORTS ---
% These packages add functionality for math symbols, formatting, etc.
\usepackage[margin=.7in]{geometry}       % For setting page margins to 0.7 inch
\usepackage{amsmath, amssymb, amsthm}   % American Mathematical Society packages for advanced math
\usepackage{graphicx}                   % For including images
\usepackage{fancyhdr}                   % For creating custom headers and footers
\usepackage[colorlinks=true, urlcolor=blue, linkcolor=blue]{hyperref} % For clickable links
\usepackage{cancel}
\usepackage{array}
\usepackage{amsfonts}
\usepackage{amsxtra}
\usepackage{epsfig}
\usepackage{wasysym}
\usepackage{relsize}
\usepackage{tikz}
\tikzset{every picture/.style={scale=1.2}}
\renewcommand{\normalsize}{\fontsize{12}{20}\selectfont}

% custom commands
\newcommand{\myauthor}{Miguel Gomez}
\newcommand{\canceling}[2]{\textcolor{red}{\cancelto{\textcolor{black}{#1}}{\textcolor{black}{#2}}}}
\newcommand{\todo}[1]{\textcolor{blue}{TODO:#1}}
% Save the original commands
\let\oldcos\cos
\let\oldsin\sin
\let\oldcosh\cosh
\let\oldsinh\sinh

% Redefine with automatic parentheses
\renewcommand{\cos}[1]{\oldcos\left(#1\right)}
\renewcommand{\sin}[1]{\oldsin\left(#1\right)}
\renewcommand{\cosh}[1]{\oldcosh\left(#1\right)}
\renewcommand{\sinh}[1]{\oldsinh\left(#1\right)}

\newcommand{\der}[2]{\frac{d#1}{d#2}}
\newcommand{\secder}[2]{\frac{d^2#1}{d#2^2}}
\newcommand{\parder}[2]{\frac{\partial#1}{\partial#2}}
\newcommand{\secparder}[2]{\frac{\partial^2#1}{\partial#2^2}}
% --- DOCUMENT & AUTHOR INFORMATION ---
\title{Worksheet \# 6}
\author{
  MATH 3160 -- Complex Variables\\
  \myauthor
}
\date{Completed: \today} 

% --- HEADER & FOOTER CONFIGURATION ---
% This section sets up the header that will appear on each page.
\pagestyle{fancy}
\fancyhf{} % Clears the default header and footer
\lhead{Math 3160 -- Worksheet \# 6} % Left side of header
\rhead{\myauthor} % Puts the author's name on the right side
\rfoot{Page \thepage} % Puts the page number on the bottom right

\begin{document}

\maketitle % This command generates the title based on the information above.

% ====================================================================
% --- START OF PROBLEMS ---
% ====================================================================

% Note: \section* creates a section heading without a number.
\section*{Problem 1}
Show that $u(x,y)$ is harmonic and find the harmonic conjugate $v(x,y)$ when:
\begin{enumerate}
  \item[(a)] $u(x,y) = 2x(1-y)$
  \item[(b)] $u(x,y) = \cos{x}\cosh{y}$ where $\cosh{y}=\frac{e^y+e^{-y}}{2}$
  \item[(c)] $u(x,y) = \frac{y}{x^2+y^2}$
  \item[(d)] $u(x,y) = \cos{x}e^y$
\end{enumerate}

A function $u(x,y):D\to\mathbb{R}$ is harmonic if:
\begin{align*}
\secparder{u(x,y)}{x} &+ \secparder{u(x,y)}{y} = 0
\end{align*}
\subsection*{(a)}
For $u(x,y) = 2x(1-y)$, we must first find the partials of $u$ and then apply these to the usual Cauchy-Riemann equations:
\begin{align*}
  \parder{u}{x} &= 2(1-y)\\
  \text{By Cauchy-Riemann, }&\text{this must equal} \quad \parder{v}{y} \\
  \parder{v}{y} &= 2(1-y)\\
\end{align*}
Given this expression, we must integrate so that we have something that would give the result when evaluated as $\parder{v}{y}$.
\begin{align*}
  \parder{C}{y} &= \parder{}{y}\left(\int 2(1-y)dy\right) = 2(1-y)\\
  \therefore &\int 2(1-y)dy = C = 2\left(y-\frac{1}{2}y^2 + c\right)
\end{align*}
Additionally, we need to have the derivative of $v$ wrt $x$ to be the negative of $u$ wrt $y$
\begin{align*}
    \parder{u}{y} &= -2x\\
  \parder{v}{x} &= 2x\\
  \therefore  v(x,y) &= x^2-y^2 + 2y + c
\end{align*}

\begin{align*}
  u(x,y) &= 2x(1-y)\\
  v(x,y) &= x^2-y^2 + 2y + c\\
  \parder{u}{x} = &2(1-y) \quad \quad \parder{u}{y} = -2x\\
  \parder{v}{y} = &2(1-y)  \quad \quad \parder{v}{x} = 2x \\
\end{align*}

Using these expressions for $u$ and $v$ now satisfy the CR equations. In order for this to be harmonic, the second derivatives must cancel to 0.
\begin{align*}
  \secparder{u(x,y)}{x} &+ \secparder{u(x,y)}{y} = 0\\
  \secparder{u(x,y)}{x} &= 0\\
  \secparder{u(x,y)}{y} &= 0 \\
  \therefore &u(x,y) \text{ is harmonic and }\\
  v(x,y) = x^2-y^2 + 2y + c &\text{ is the harmonic conjugate.}
\end{align*}
\subsection*{(b)}
Once again, we will follow the same steps as above but I will only show the work now for the rest to save space. $u(x,y) = \cos{x}\cosh{y}$.
\begin{align*}
  \parder{u}{x} = -\sin{x}&\cosh{y} \quad \quad \parder{u}{y} = \cos{x}\sinh{y}\\
  \parder{v}{y} &= \parder{u}{x} = -\sin{x}\cosh{y}\\
  \parder{u}{y} &=-\parder{v}{x} = -\cos{x}\sinh{y}\\
  \secparder{u(x,y)}{x} &+ \secparder{u(x,y)}{y} = 0\\
-\cos{x}\cosh{y} &+ \cos{x}\cosh{y} = 0 \\
  \therefore v(x,y) & \text{ is harmonic and}\\
  \therefore v(x,y) &= -\sin{x}\sinh{y} + c \quad \text{ is the HC}
\end{align*}
\subsection*{(c)}
\begin{align*}
  u(x,y) &= \frac{y}{x^2+y^2} = y(x^2+y^2)^{-1}\\
  \parder{u}{x} &= (-1)y(x^2+y^2)^{-2}(2x)\\
  \parder{u}{y} &= \parder{(fg)}{y} = f'g+g'f\\
  f &= y \quad \quad f' = 1\\
  g &= (x^2+y^2)^{-1} \quad \quad g' = (-1)(x^2+y^2)^{-2}(2y)\\
  f'g+g'f &= (1)(x^2+y^2)^{-1} + y(-1)(x^2+y^2)^{-2}(2y) \\
         &= (x^2+y^2)^{-1} - 2y^2(x^2+y^2)^{-2}\\
  \parder{u}{x} &= \parder{v}{y} = -2xy(x^2+y^2)^{-2}
\end{align*}
This appears to be a form that looks to be related to a complex number divided by its magnitude. We can try to guess the function from here:
\begin{align*}
  \frac{z}{|z|^2} &= \frac{x+iy}{x^2+y^2} = \frac{x}{x^2+y^2} + i\frac{y}{x^2+y^2}
\end{align*}
Rotating it by $i$:
\begin{align*}
  \frac{-iz}{|z|^2} &= -i\frac{x+iy}{x^2+y^2} = -i\frac{x}{x^2+y^2} -i^2\frac{y}{x^2+y^2}\\
  &= \frac{y}{x^2+y^2}-i\frac{x}{x^2+y^2}
\end{align*}
\todo{actually work this out. and show this is correct}
\subsection*{(d)}
\begin{align*}
  u(x,y) &= \cos{x}e^y
\end{align*}
This one is the real part of a complex number defined as $f = e^z$ where $z = y+ix$:
\begin{align*}
  f = e^y(\cos{x}+i\sin{x})
\end{align*}
using this we can infer that the HC is the following:
\begin{align*}
  v(x,y) = \sin{x}e^y
\end{align*}
\todo{actually work this out.}
\newpage
\section*{Problem 2}
Suppose that $v$ is a harmonic conjugate of $u$ and $u$ is a harmonic conjugate of $v$ on some domain $D$. Show that $u,v$ must then be constant on $D$. (Hint: show that all partial derivatives of $u,v$ vanish on $D$)


\end{document}
