% --------------------------------------------------------------------
% LaTeX Template for Math Worksheets
% --------------------------------------------------------------------
\documentclass{article}

% --- PACKAGE IMPORTS ---
% These packages add functionality for math symbols, formatting, etc.
\usepackage[margin=.7in]{geometry}       % For setting page margins to 0.7 inch
\usepackage{amsmath, amssymb, amsthm}   % American Mathematical Society packages for advanced math
\usepackage{graphicx}                   % For including images
\usepackage{fancyhdr}                   % For creating custom headers and footers
\usepackage[colorlinks=true, urlcolor=blue, linkcolor=blue]{hyperref} % For clickable links
\usepackage{cancel}
\usepackage{array}
\usepackage{amsfonts}
\usepackage{amsxtra}
\usepackage{epsfig}
\usepackage{wasysym}
\usepackage{relsize}
\usepackage{tikz}
\tikzset{every picture/.style={scale=1.2}}
\renewcommand{\normalsize}{\fontsize{12}{20}\selectfont}

% custom commands
\newcommand{\myauthor}{Miguel Gomez}
\newcommand{\canceling}[2]{\textcolor{red}{\cancelto{\textcolor{black}{#1}}{\textcolor{black}{#2}}}}
\newcommand{\todo}[1]{\textcolor{blue}{TODO:#1}}
% Save the original commands
\let\oldcos\cos
\let\oldsin\sin
\let\oldcosh\cosh
\let\oldsinh\sinh

% Redefine with automatic parentheses
\renewcommand{\cos}[1]{\oldcos\left(#1\right)}
\renewcommand{\sin}[1]{\oldsin\left(#1\right)}
\renewcommand{\cosh}[1]{\oldcosh\left(#1\right)}
\renewcommand{\sinh}[1]{\oldsinh\left(#1\right)}

\newcommand{\der}[2]{\frac{d#1}{d#2}}
\newcommand{\secder}[2]{\frac{d^2#1}{d#2^2}}
\newcommand{\parder}[2]{\frac{\partial#1}{\partial#2}}
\newcommand{\secparder}[2]{\frac{\partial^2#1}{\partial#2^2}}

% --- DOCUMENT & AUTHOR INFORMATION ---
\title{Worksheet \# 7}
\author{
  MATH 3160 -- Complex Variables\\
  \myauthor
}
\date{Completed: \today} 

% --- HEADER & FOOTER CONFIGURATION ---
% This section sets up the header that will appear on each page.
\pagestyle{fancy}
\fancyhf{} % Clears the default header and footer
\lhead{Math 3160 -- Worksheet \# 7} % Left side of header
\rhead{\myauthor} % Puts the author's name on the right side
\rfoot{Page \thepage} % Puts the page number on the bottom right

\begin{document}

\maketitle % This command generates the title based on the information above.

% ====================================================================
% --- START OF PROBLEMS ---
% ====================================================================

% Note: \section* creates a section heading without a number.
\section*{Problem 1}
suppose $l_1(z)$ and $l_2(z)$ denote two different branches of log. What can you say about the function $l_1(z)-l_2(z)$?
How many different values can the difference take? Use this to conclude that $\der{}{z} (l(z)) = \frac{1}{z}$ for any branch of log (away from its branch cut).
\vspace{.5cm}
\hrule % Adds a horizontal line to separate problems.
\vspace{.5cm}

This seems similar to the argument made in class about the arg of a complex number and the value it could take in a branch. Some similar reasoning will need to be made here to show that the result must be $\frac{1}{z}$.
\begin{align*}
  
\end{align*}

\vspace{1cm}
\hrule % Adds a horizontal line to separate problems.

\newpage
\section*{Problem 2}
Check that for any $c_1\ c_2 \in \mathbb{C}$, and any $ 0 \neq z \in \mathbb{C}$, we have $z^{(c_1+c_2)} = z^{c_1} z^{c_2}$ using the similar rule for the exponential function.
\vspace{.5cm} % Space for work
\hrule
\vspace{.5cm} % Space for work

\begin{align*}
  c_1\text{ , } c_2 &= x_1 + iy_1 \quad\text{,}\quad x_2 + iy_2 \\
  w = c_1+c_2 &= x_1 + iy_1 + x_2 + iy_2 = x_1 + x_2 + iy_1 + iy_2\\
  z^w &= z^{(c_1+c_2)} = z^{x_1 + iy_1 + x_2 + iy_2} = z^{x_1 + x_2 + iy_1 + iy_2}\\
                    &= z^{x_1}z^{ x_2 }z^{iy_1 }z^{ iy_2} = z^{x_1}z^{iy_1}z^{x_2}z^{iy_2} \\
                    &= z^{x_1+iy_1}z^{x_2+iy_2} = z^{c_1}z^{c_2}\\
                    \therefore z^{(c_1+c_2)} &= z^{c_1}z^{c_2}\ \   \forall x\text{,}y \in \mathbb{R} | z \neq 0 
\end{align*}

\newpage
\section*{Problem 3}
Using the chain rule, show that $\der{}{z} z^c = cz^{c−1}$.

\vspace{1cm} % Space for work

\hrule

% Add more problems as needed...
% \newpage
% \section*{Problem 4}
% 
% \hrule

\end{document}
