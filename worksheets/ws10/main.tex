% --------------------------------------------------------------------
% LaTeX Template for Math Worksheets
% --------------------------------------------------------------------
\documentclass{article}

% --- PACKAGE IMPORTS ---
% These packages add functionality for math symbols, formatting, etc.
\usepackage[margin=.7in]{geometry}       % For setting page margins to 0.7 inch
\usepackage{amsmath, amssymb, amsthm}   % American Mathematical Society packages for advanced math
\usepackage{graphicx}                   % For including images
\usepackage{fancyhdr}                   % For creating custom headers and footers
\usepackage[colorlinks=true, urlcolor=blue, linkcolor=blue]{hyperref} % For clickable links
\usepackage{cancel}
\usepackage{array}
\usepackage{amsfonts}
\usepackage{amsxtra}
\usepackage{epsfig}
\usepackage{wasysym}
\usepackage{relsize}
\usepackage{tikz}
\tikzset{every picture/.style={scale=1.2}}
\renewcommand{\normalsize}{\fontsize{12}{20}\selectfont}

% custom commands
\newcommand{\myauthor}{Miguel Gomez}
\newcommand{\canceling}[2]{\textcolor{red}{\cancelto{\textcolor{black}{#1}}{\textcolor{black}{#2}}}}
\newcommand{\todo}[1]{\textcolor{blue}{TODO:#1}}
% Save the original commands
\let\oldcos\cos
\let\oldsin\sin
\let\oldcosh\cosh
\let\oldsinh\sinh

% Redefine with automatic parentheses
\renewcommand{\cos}[1]{\oldcos\left(#1\right)}
\renewcommand{\sin}[1]{\oldsin\left(#1\right)}
\renewcommand{\cosh}[1]{\oldcosh\left(#1\right)}
\renewcommand{\sinh}[1]{\oldsinh\left(#1\right)}

\newcommand{\der}[2]{\frac{d#1}{d#2}}
\newcommand{\secder}[2]{\frac{d^2#1}{d#2^2}}
\newcommand{\parder}[2]{\frac{\partial#1}{\partial#2}}
\newcommand{\secparder}[2]{\frac{\partial^2#1}{\partial#2^2}}

% --- DOCUMENT & AUTHOR INFORMATION ---
\title{Worksheet \# 10}
\author{
  MATH 3160 -- Complex Variables\\
  \myauthor
}
\date{Completed: \today} 

% --- HEADER & FOOTER CONFIGURATION ---
% This section sets up the header that will appear on each page.
\pagestyle{fancy}
\fancyhf{} % Clears the default header and footer
\lhead{Math 3160 -- Worksheet \# 10} % Left side of header
\rhead{\myauthor} % Puts the author's name on the right side
\rfoot{Page \thepage} % Puts the page number on the bottom right

\begin{document}

\maketitle % This command generates the title based on the information above.

% ====================================================================
% --- START OF PROBLEMS ---
% ====================================================================

% Note: \section* creates a section heading without a number.
\section*{Problem 1}
Use the Taylor series of $\sin{z}$, $\cos{z}$, and $e^{z}$ at $z=0$ to prove Euler's formula:
\begin{align*}
e^{iz} = \cos{z} + i\sin{z}
\end{align*}

\vspace{.5cm}
\hrule % Adds a horizontal line to separate problems.
\vspace{.5cm}
Using the known expression for the expansion of the exponential shown above:
\begin{align*}
  e^x &= \sum_{n=0}^{\infty}\frac{x^n}{n!}\\
  e^{iz} &= \sum_{n=0}^{\infty}\frac{(iz)^n}{n!}\\
  = 1 + \frac{(iz)^{1}}{1!} + \frac{(iz)^{2}}{2!} + \frac{(iz)^{3}}{3!} &+ \frac{(iz)^{4}}{4!} + \frac{(iz)^{5}}{5!} + \frac{(iz)^{6}}{6!} + ... \\
  = 1 + i(z)^{1} + i^2\frac{(z)^{2}}{2!} + i^3\frac{(z)^{3}}{3!} &+ i^4\frac{(z)^{4}}{4!} + i^5\frac{(z)^{5}}{5!} + i^6\frac{(z)^{6}}{6!} + ... \\
\end{align*}
The resulting pattern shows that we have $i$ in many of the expressions, but we can reduce all the products of $i$ with itself to the following.
\begin{align*}
  i^{0} = i^{4} = i^{4n} &= 1\\
  i^{1} = i^{5} = i^{4n + 1} &= i\\
  i^{2} = i^{6} = i^{4n + 2} &= -1\\
  i^{3} = i^{7} = i^{4n + 3} &= -i
\end{align*}
This shows that the expansion can be reduced to either have a factor of $i$ or not. Therefore, we can reduce the expansion further:
\begin{align*}
  = 1 + i(z)^{1} + (-1)\frac{(z)^{2}}{2!} -i\frac{(z)^{3}}{3!} &+ (1)\frac{(z)^{4}}{4!} + i\frac{(z)^{5}}{5!} - (1)\frac{(z)^{6}}{6!} + ... \\
  = 1 + i(z)^{1} - \frac{(z)^{2}}{2!} - i\frac{(z)^{3}}{3!} &+ \frac{(z)^{4}}{4!} + i\frac{(z)^{5}}{5!} - (1)\frac{(z)^{6}}{6!} + ... \\
  = 1 - \frac{(z)^{2}}{2!}  + \frac{(z)^{4}}{4!} - (1)\frac{(z)^{6}}{6!} &+ i(z)^{1} - i\frac{(z)^{3}}{3!} + i\frac{(z)^{5}}{5!} ... \\
  = \left(1 - \frac{\left(z\right)^{2}}{2!}  + \frac{\left(z\right)^{4}}{4!} - \frac{\left(z\right)^{6}}{6!} + ...\right) &+ i\left(\left(z\right)^{1} - \frac{\left(z\right)^{3}}{3!} + \frac{\left(z\right)^{5}}{5!} ...\right) \\
\end{align*}
The expansion of $\sin{z}$ and $\cos{z}$:
\begin{align*}
  \sin{z} &= \sum_{n=0}^{\infty}(-1)^{n}\frac{z^{2n+1}}{{2n+1}!}\\
  = (z)^{1} -\frac{(z)^{3}}{3!} &+ \frac{(z)^{5}}{5!} - \frac{(z)^{7}}{7!} + ... \\
  \cos{z} &= \sum_{n=0}^{\infty}(-1)^{n}\frac{z^{2n}}{{2n}!}\\
  = 1 - \frac{(z)^{2}}{2!} &+ \frac{(z)^{4}}{4!} - \frac{(z)^{6}}{6!} + ... \\
\end{align*}
Notice how the $\sin{z}$ and $\cos{z}$ expansions are exactly what we see in the expression above when expanding out and grouping the expression of $e^{iz}$
\begin{align*}
  \therefore e^{iz} &=  \sum_{n=0}^{\infty}(-1)^{n}\frac{z^{2n}}{{2n}!} + i\sum_{n=0}^{\infty}(-1)^{n}\frac{z^{2n+1}}{{2n+1}!}\\
   &= \cos{z} + i\sin{z}
\end{align*}
\newpage
\section*{Problem 2}
Find $f^{(10)}(3)$ for $f(z)$ as given below. You can leave the answer in terms of powers, factorials, etc. 
\begin{align*}
  f(z) &= \sum_{n=0}^{\infty}n^2(z-3)^n
\end{align*}

\vspace{.5cm} % Space for work
\hrule
\vspace{.5cm} % Space for work

To do this, we can employ the following expressions:
\begin{align*}
  f(z) &= \sum_{n=0}^{\infty}a_n(z-z_0)^n\\
  f^{(n)}(z) &= n!a_n\\
  a_n &= \frac{f^{n}(z_0)}{n!}
\end{align*}
So any term that would be present at the 10th derivative summation greater than n would disappear from being a constant or having a factor of $(z-3)$. Meaning we only get the final term being:
\begin{align*}
  f^{(10)}(3) &= 10!a_n\\
  a_n &= n^2 = 10^2 = 100\\
  \therefore f^{(10)}(3) &= 100\cdot 10!
\end{align*}
\newpage
\section*{Problem 3}
List the first four nonzero terms of the Taylor series for $f(z) = \cos{3z+2}$ centered at $z=-\frac{2}{3}$.

\vspace{.5cm} % Space for work
\hrule
\vspace{.5cm} % Space for work
Here, we need to change the expression to look like $a_n(z-z_0)$:
\begin{align*}
  \cos{3z+2} &= \cos{3\left(z+\frac{2}{3}\right)}\\
\end{align*}
expanding this out with the expansion we showed before:
\begin{align*}
    \cos{z} &= \sum_{n=0}^{\infty}(-1)^{n}\frac{z^{2n}}{{2n}!}\\
  = 1 - \frac{(z)^{2}}{2!} &+ \frac{(z)^{4}}{4!} - \frac{(z)^{6}}{6!} + ... \\
\end{align*}
This is already the first four terms, we only need to replace the $z$ with our expression:
\begin{align*}
  1 - \frac{(3(z+\frac{2}{3}))^{2}}{2!} &+ \frac{(3(z+\frac{2}{3}))^{4}}{4!} - \frac{(3(z+\frac{2}{3}))^{6}}{6!}
\end{align*}
% Add more problems as needed...
% \newpage
% \section*{Problem 4}
% 
% \hrule

\end{document}
