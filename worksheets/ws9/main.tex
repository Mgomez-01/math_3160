% --------------------------------------------------------------------
% LaTeX Template for Math Worksheets
% --------------------------------------------------------------------
\documentclass{article}

% --- PACKAGE IMPORTS ---
% These packages add functionality for math symbols, formatting, etc.
\usepackage[margin=.7in]{geometry}       % For setting page margins to 0.7 inch
\usepackage{amsmath, amssymb, amsthm}   % American Mathematical Society packages for advanced math
\usepackage{graphicx}                   % For including images
\usepackage{fancyhdr}                   % For creating custom headers and footers
\usepackage[colorlinks=true, urlcolor=blue, linkcolor=blue]{hyperref} % For clickable links
\usepackage{cancel}
\usepackage{array}
\usepackage{amsfonts}
\usepackage{amsxtra}
\usepackage{epsfig}
\usepackage{wasysym}
\usepackage{relsize}
\usepackage{tikz}
\tikzset{every picture/.style={scale=1.2}}
\usetikzlibrary{decorations.markings} 
\renewcommand{\normalsize}{\fontsize{12}{20}\selectfont}

% custom commands
\newcommand{\myauthor}{Miguel Gomez}
\newcommand{\canceling}[2]{\textcolor{red}{\cancelto{\textcolor{black}{#1}}{\textcolor{black}{#2}}}}
\newcommand{\todo}[1]{\textcolor{blue}{TODO:#1}}
% Save the original commands
\let\oldcos\cos
\let\oldsin\sin
\let\oldcosh\cosh
\let\oldsinh\sinh

% Redefine with automatic parentheses
\renewcommand{\cos}[1]{\oldcos\left(#1\right)}
\renewcommand{\sin}[1]{\oldsin\left(#1\right)}
\renewcommand{\cosh}[1]{\oldcosh\left(#1\right)}
\renewcommand{\sinh}[1]{\oldsinh\left(#1\right)}

\newcommand{\der}[2]{\frac{d#1}{d#2}}
\newcommand{\secder}[2]{\frac{d^2#1}{d#2^2}}
\newcommand{\parder}[2]{\frac{\partial#1}{\partial#2}}
\newcommand{\secparder}[2]{\frac{\partial^2#1}{\partial#2^2}}

% --- DOCUMENT & AUTHOR INFORMATION ---
\title{Worksheet \# 9}
\author{
  MATH 3160 -- Complex Variables\\
  \myauthor
}
\date{Completed: \today} 

% --- HEADER & FOOTER CONFIGURATION ---
% This section sets up the header that will appear on each page.
\pagestyle{fancy}
\fancyhf{} % Clears the default header and footer
\lhead{Math 3160 -- Worksheet \# 9} % Left side of header
\rhead{\myauthor} % Puts the author's name on the right side
\rfoot{Page \thepage} % Puts the page number on the bottom right

\begin{document}

\maketitle % This command generates the title based on the information above.

% ====================================================================
% --- START OF PROBLEMS ---
% ====================================================================

% Note: \section* creates a section heading without a number.
\section*{Problem 1}
let C be the contour shown below, traversed counter-clockwise from the blue point to the red. I have reconstructed this image from the worksheet and am confident this is similar to how it was done, but I must admit I am making an assumption about the structure. 
\begin{center}
\begin{tikzpicture}
    % Draw grid
    \draw[gray!30, very thin] (-2.5,-2.5) grid (2.5,2.5);
    
    % Draw axes
    \draw[->] (-2.5,0) -- (2.5,0) node[right] {};
    \draw[->] (0,-2.5) -- (0,2.5) node[above] {};
    
    % Add axis labels
    \foreach \x in {-2,-1,1,2}
        \draw (\x,0.1) -- (\x,-0.1) node[below] {\x};
    \foreach \y in {-2,-1,1,2}
        \draw (0.1,\y) -- (-0.1,\y) node[left] {\y};
    
    % Draw the wavy circle with arrow decorations
    \draw[thick] plot[domain=-225:45, samples=300, smooth] 
        ({\x}: {sqrt(2) + 0.05*sin(40*\x)});
        % Draw the wavy circle with arrow decorations
    \draw[gray!30, decoration={markings, 
        mark=at position 0.15 with {\arrow[black]{>}},
        mark=at position 0.35 with {\arrow[black]{>}},
        mark=at position 0.55 with {\arrow[black]{>}},
        mark=at position 0.75 with {\arrow[black]{>}},
        mark=at position 0.95 with {\arrow[black]{>}}},
        postaction={decorate}] 
        plot[domain=-225:45, samples=300, smooth] 
        ({\x}: {sqrt(2.8)});


    % Add dots at start and end points
    \fill[red] (45:{sqrt(2)}) circle (2pt);
    \fill[blue] (-225:{sqrt(2)}) circle (2pt);
\end{tikzpicture}
\end{center}
\vspace{0.5cm} % Add space for additional work if needed

% steal fig from the latex given on canvas

find $\int_C \frac{1}{z}dz$ (Hint: Consider a new branch of the logarithm function by $\log{(re^{i\theta})} = \ln{(r)} + i\theta$, where $-3\pi/2 < \theta \leq \pi/2$, and check that this is an anti-derivative of $1/z$.)
\vspace{0.5cm}
\hrule % Adds a horizontal line to separate problems.
\vspace{0.5cm}
I get the hint, but I saw this from the start and wanted to work it out. In checking this path, we know that the result should have a value because it is not a closed path. Parametrizing this path works as follows given the diagram. placing a circle of radius $\sqrt{2}$ cuts through the sinusoid and it oscillates around it.  
\begin{align*}
  z &= (\sqrt{2}+A\sin{\omega\theta})e^{i\theta} \\
  z_0 &= \sqrt{2}e^{-i\frac{5\pi}{4}} \quad A\sin{-\omega\frac{5\pi}{4}} = 0 \\
  z_f &= \sqrt{2}e^{i\frac{\pi}{4}}  \quad A\sin{\omega\frac{\pi}{4}} = 0\\
  r(t) &= \sqrt{2}+A\sin{\omega\theta(t)}
\end{align*}

The path has a sinusoidal signal in superposition such that there is a change $A\sin{\omega\theta}$ in the magnitude of $z$. Instead of writing out so much, we can continue by treating this more generally:

\begin{align*}
  z(t) &= r(t)e^{i\theta(t)} \\
  z'(t) &= r'(t)e^{i\theta(t)}+r(t)(i\theta'(t))e^{i\theta(t)}\\
  \int_{\gamma}f(z)dz &= \int_{t_0}^{t_1}f(z(t))z'(t)dt \\
  \int_{\gamma}f(z)dz &= \int_{t_0}^{t_1}\frac{1}{r(t)e^{i\theta(t)}}[r'(t)e^{i\theta(t)}+r(t)(i\theta'(t))e^{i\theta(t)}]dt \\
&= \int_{t_0}^{t_1}\frac{e^{i\theta(t)}}{r(t)e^{i\theta(t)}}[r'(t)+r(t)(i\theta'(t))]dt \\
       &= \int_{t_0}^{t_1}\left[\frac{r'(t)}{r(t)}+i\theta'(t)\right]dt \\
  &= \int_{t_0}^{t_1}\frac{r'(t)}{r(t)}dt +i\int_{t_0}^{t_1} \theta'(t)dt \\
  &= \ln{(r(t))}|_{t_0}^{t_1} +i \theta(t)|_{t_0}^{t_1} \\
  &= (\ln{(r(t_1))} - \ln{(r(t_0))}) + i (\theta(t_1)-\theta(t_0)) \\
\end{align*}
Now, we can see that no matter the function $r(t)$, we get the final expressions by recognizing that the  sinusoidal signal for the magnitude has the same value at $t_0$ and $t_1$, then $r(t_1) = r(t_0)$, and therefore $\ln{(r(t_1))} = \ln{(r(t_0))}$. This then leaves us with the final expression:
\begin{align*}
\int_{\gamma}f(z)dz &=  i (\theta(t_1)-\theta(t_0))
\end{align*}
This shows that the integral value only depends on the angle difference and would not change for any radius used. Therefore, since the angular difference is $3/4$ of the unit circle, and we know the integral of $1/z$ is $2\pi i$, this integral is therefore $\frac{3\pi}{2} i$. Which is nice because it confirms the given hint and why it works as an antiderivative.
\newpage
\section*{Problem 2}
Show that $\int_C f(z)dz = 0$ for $C$ the unit circle and :
\begin{enumerate}
  \item[(i)] $f(z) = \frac{z^2}{z+3}$
  \item[(ii)] $f(z) = \frac{1}{z^2 + 2z + 2}$
\end{enumerate}
\subsection*{(i)}
Since we are evaluating with $C$ within the unit circle, any point which lies outside of the unit circle does not matter for our evaluation as we only need the curve and its interior to be a simply connected domain $D$. in the denominator, we see that we have $z+3$, meaning that it only becomes 0 if $z$ is $-3$. So the point $z = -3$ in the complex plane will give a divide by zero issue. Since $|−3| = 3 > 1$, the unit circle only contains points where the magnitude of $z$ is less than or equal to $1$, meaning it is analytic inside and on the unit circle.

$\therefore$ the C-G theorem holds and we have a path with a simply connected interior region with the same starting and ending point whose integral evaluates to $0$. 

\subsection*{(ii)}
In this problem, we have a similar result as the denominator is $0$ only where $z = -1 \pm i$ given the factoring of the denominator. Notice that the magnitude of $z$ for these points will be $\sqrt{2}$, and therefore the points are also outside of the unit circle. With $\sqrt{2} > 1$ and the unit circle only contains points where $|z| \leq 1$, then we again have an integral that evaluates to $0$. 

\vspace{0.5cm} % Space for work

\hrule

\newpage
\section*{Problem 3}
Let $C_1$ denote the positively oriented boundary of the square whose sides lie along the lines $x = \pm 1$, and $y=\pm i$, and let $C_2$ denote the positively oriented circle $|z| = 4$. Explain why:
\begin{align*}
  \int_{C_1}f(z)dz &= \int_{C_2}f(z)dz
\end{align*}
when
\begin{enumerate}
  \item[(a)] $f(z) = \frac{1}{2x^2+1}$
  \item[(b)] $f(z) = \frac{z+2}{sin(z/2)}$
  \item[(c)] $f(z) = \frac{z}{1-e^z}$
\end{enumerate}

\vspace{0.5cm} % Space for work

\hrule

% Add more problems as needed...
% \newpage
% \section*{Problem 4}
% 
% \hrule

\end{document}

%%% Local Variables:
%%% mode: latex
%%% TeX-master: t
%%% End:
