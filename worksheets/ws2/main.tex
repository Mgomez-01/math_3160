% --------------------------------------------------------------------
% LaTeX Template for Math Worksheets
% --------------------------------------------------------------------

\documentclass{article}

% --- PACKAGE IMPORTS ---
\usepackage[margin=1in]{geometry}
\usepackage{amsmath, amssymb, amsthm}
\usepackage{graphicx}
\usepackage{fancyhdr}
\usepackage{xcolor}   % Required for \textcolor
\usepackage{cancel}   % Required for \cancelto
\usepackage[colorlinks=true, urlcolor=blue, linkcolor=blue]{hyperref}

% --- CUSTOM COMMANDS & OPERATORS ---
% Defines a command \cancelled{new}{old}
\newcommand{\canceling}[2]{\textcolor{red}{\cancelto{\textcolor{black}{#1}}{\textcolor{black}{#2}}}}

% Defines a proper "Arg" operator for complex analysis
\DeclareMathOperator{\Arg}{Arg}

% --- DOCUMENT & AUTHOR INFORMATION ---
\newcommand{\myauthor}{Miguel Gomez} % Define a custom command for your name
\title{Worksheet 2}
\author{
  MATH 3160\\
  \myauthor
}
\date{Completed: \today} % The due date is from the source 

% --- HEADER & FOOTER CONFIGURATION ---
\pagestyle{fancy}
\fancyhf{}
\chead{Math 3160 -- Worksheet 2}
\rhead{\myauthor} % Use the same custom command for the header
\rfoot{Page \thepage}

\begin{document}

\maketitle

% ====================================================================
% --- START OF PROBLEMS ---
% ====================================================================

\section*{Problem 1}
Reduce each of these to a real number

\hrule
\begin{enumerate}
\item[(a)] $ \frac{1+2i}{3-4i} + \frac{2-i}{5i}$
  \begin{align*}
    \frac{1+2i}{3-4i} &+ \frac{2-i}{5i} = \\
    \frac{(1+2i)(3+4i)}{(3-4i)(3+4i)} &+ \frac{(2-i)(-5i)}{(5i)(-5i)} = \\
    \frac{(1+2i)(3+4i)}{9-16i^2} &+ \frac{(2-i)(-5i)}{-25i^2} = \\
    \frac{(3+4i+6i+8\canceling{-1}{i^2})}{9-16\canceling{-1}{i^2}} &+ \frac{(-10i+5\canceling{-1}{i^2})}{-25\canceling{-1}{i^2}} = \\
    \frac{(3+4i+6i-8)}{25} &+ \frac{(-10i-5)}{25} = \\
    \frac{(3+4i+6i-8)}{25} &+ \frac{(-10i-5)}{25} = \\
    \frac{(-5+10i)}{25} &+ \frac{(-10i-5)}{25} = \\
    \frac{(-5-5+10i -10i)}{25} = -\frac{2}{5}\\
  \end{align*}
\item[(b)] $ \frac{5i}{(1-i)(2-i)(3-i)}$
  \begin{align*}
    \frac{5i}{(1-i)(2-i)(3-i)} &= \frac{5i}{(2-i-2i+i^2)(3-i)} = \\
    \frac{5i}{(1-3i)(3-i)} &= \frac{5i}{(1-3i)(3-i)} =\\
    \frac{5i}{-10i} &= -\frac{1}{2}
  \end{align*}
\end{enumerate}

\newpage
\section*{Problem 2}
Find the principal argument $\Arg z$ when..
\hrule
preliminary necessary expressions:
  \begin{align*}
    \text{arg}(z) &= \text{Arg}(z) + 2\cdot pi\cdot k\ \ \ ;k\in \mathbb{Z} \\
    e^{i\theta} &= \cos\theta + i\sin{\theta}\\
    -\pi & < \theta \le \pi 
  \end{align*}

\begin{enumerate}
    \item[(a)] $z=\frac{-2}{1+\sqrt{3}i}$ 
    \vspace{4cm}
    
    \item[(b)] $z=\frac{2i}{i-1}$ 
    \vspace{4cm}
    
    \item[(c)] $z=(\sqrt{3}-i)^{6}$ 
    \vspace{4cm}
\end{enumerate}

\hrule

\section*{Problem 3}
For the next few questions write the individual factors on the left in exponential form, perform the needed operations on complex numbers, and finally change back to rectangular coordinates \textit{Show that}: 

\begin{enumerate}
    \item[(a)] $i(1-\sqrt{3}i)(\sqrt{3}+i)=2(1+\sqrt{3}i)$ 
    \vspace{5cm}
    
    \item[(b)] $(\sqrt{3}+i)^{6}=-64$ 
    \vspace{5cm}
    
    \item[(c)] $(1+\sqrt{3}i)^{-10}=2^{-11}(-1+\sqrt{3}i)$ 
    \vspace{5cm}
\end{enumerate}

\hrule

\section*{Problem 4}
Use exponential form to find $(1-i)^{5}$

\vspace{6cm}

\end{document}
%%% Local Variables:
%%% mode: latex
%%% TeX-master: t
%%% End:
