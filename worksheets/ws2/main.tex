% --------------------------------------------------------------------
% LaTeX Template for Math Worksheets
% --------------------------------------------------------------------

\documentclass{article}

% --- PACKAGE IMPORTS ---
\usepackage[margin=1in]{geometry}
\usepackage{amsmath, amssymb, amsthm}
\usepackage{graphicx}
\usepackage{fancyhdr}
\usepackage{xcolor}   % Required for \textcolor
\usepackage{cancel}   % Required for \cancelto
\usepackage[colorlinks=true, urlcolor=blue, linkcolor=blue]{hyperref}

% --- CUSTOM COMMANDS & OPERATORS ---
% Defines a command \cancelled{new}{old}
\newcommand{\canceling}[2]{\textcolor{red}{\cancelto{\textcolor{black}{#1}}{\textcolor{black}{#2}}}}

% Defines a proper "Arg" operator for complex analysis
\DeclareMathOperator{\Arg}{Arg}

% --- DOCUMENT & AUTHOR INFORMATION ---
\newcommand{\myauthor}{Miguel Gomez} % Define a custom command for your name
\title{Worksheet 2}
\author{
  MATH 3160\\
  \myauthor
}
\date{Completed: \today} % The due date is from the source 

% --- HEADER & FOOTER CONFIGURATION ---
\pagestyle{fancy}
\fancyhf{}
\chead{Math 3160 -- Worksheet 2}
\rhead{\myauthor} % Use the same custom command for the header
\rfoot{Page \thepage}

\begin{document}

\maketitle

% ====================================================================
% --- START OF PROBLEMS ---
% ====================================================================

\section*{Problem 1}
Reduce each of these to a real number

\hrule
\begin{enumerate}
\item[(a)] $ \frac{1+2i}{3-4i} + \frac{2-i}{5i}$
  \begin{align*}
    \frac{1+2i}{3-4i} &+ \frac{2-i}{5i} = \\
    \frac{(1+2i)(3+4i)}{(3-4i)(3+4i)} &+ \frac{(2-i)(-5i)}{(5i)(-5i)} = \\
    \frac{(1+2i)(3+4i)}{9-16i^2} &+ \frac{(2-i)(-5i)}{-25i^2} = \\
    \frac{(3+4i+6i+8\canceling{-1}{i^2})}{9-16\canceling{-1}{i^2}} &+ \frac{(-10i+5\canceling{-1}{i^2})}{-25\canceling{-1}{i^2}} = \\
    \frac{(3+4i+6i-8)}{25} &+ \frac{(-10i-5)}{25} = \\
    \frac{(3+4i+6i-8)}{25} &+ \frac{(-10i-5)}{25} = \\
    \frac{(-5+10i)}{25} &+ \frac{(-10i-5)}{25} = \\
    \frac{(-5-5+10i -10i)}{25} = \boxed{-\frac{2}{5}}\\
  \end{align*}
\item[(b)] $ \frac{5i}{(1-i)(2-i)(3-i)}$
  \begin{align*}
    \frac{5i}{(1-i)(2-i)(3-i)} &= \frac{5i}{(2-i-2i+i^2)(3-i)} = \\
    \frac{5i}{(1-3i)(3-i)} &= \frac{5i}{(1-3i)(3-i)} =\\
    \frac{5i}{-10i} &= \boxed{-\frac{1}{2}}
  \end{align*}
\end{enumerate}

\newpage
\section*{Problem 2}
Find the principal argument $\Arg z$ when..
\hrule
preliminary necessary expressions:
  \begin{align*}
    \text{arg}(z) &= \text{Arg}(z) + 2\cdot pi\cdot k\ \ \ ;k\in \mathbb{Z} \\
    e^{i\theta} &= \cos\theta + i\sin{\theta}\\
    -\pi & < \theta \le \pi 
  \end{align*}

\begin{enumerate}
    \item[(a)] $z=\frac{-2}{1+\sqrt{3}i}$ 
      \begin{align*}
        \frac{-2}{1+\sqrt{3}i} & = \frac{-2}{1+\sqrt{3}i} \frac{1-\sqrt{3}i}{1-\sqrt{3}i} =\\
        \frac{-2(1-\sqrt{3}i)}{1\canceling{0}{-\sqrt{3}i + \sqrt{3}i }+ \sqrt{3}^2\canceling{-1}{i^2}}\ \ \  &=  \frac{-2+2\sqrt{3}i}{1-3}\\
        2\left(-\frac{1}{2} + \frac{\sqrt{3}}{2}i\right) &= 2e^{i\frac{2\pi}{3}}\\
        \therefore Arg{z} &= \boxed{\frac{2\pi}{3}}
      \end{align*}
    \item[(b)] $z=\frac{2i}{i-1}$ 
      \begin{align*}
        \frac{2i}{i-1} &= \frac{2i(-i-1)}{(i-1)(-i-1)} = \frac{2i(-1-i)}{(-1 + i)(-1-i)} = \\
        \frac{(-2i-2i^2)}{1+i-i-i^2} &= \frac{(-2i-2\canceling{-1}{i^2})}{1\canceling{0}{+i-i}-\canceling{-1}{i^2}} \ \ \ = \frac{2-2i}{2} = 1-i = \sqrt{2}e^{-i\frac{\pi}{4}}\\
        \therefore \text{Arg}(z) &= \boxed{-\frac{\pi}{4}}  
      \end{align*}
    \item[(c)] $z=(\sqrt{3}-i)^{6}$
      
      For this one, we need to first include a factor of $2^6$ and divide by it as well to bring a half into the parentheses.
      \begin{align*}
        (\sqrt{3}-i)^{6} &= 2^6\left(\frac{\sqrt{3}}{2}-\frac{1}{2}i\right)^{6} \\
        2^6\left(e^{-i\frac{\pi}{6}}\right)^6 &= 2^6e^{-i\pi} \\
        \therefore \text{Arg}(z) &= \boxed{\pi} 
      \end{align*}
    \end{enumerate}

\hrule
\newpage
\section*{Problem 3}
For the next few questions write the individual factors on the left in exponential form, perform the needed operations on complex numbers, and finally change back to rectangular coordinates \textit{Show that}: 

\begin{enumerate}
    \item[(a)] $i(1-\sqrt{3}i)(\sqrt{3}+i)=2(1+\sqrt{3}i)$ 

      To convert into exponential form, we can add factors to get us the correct exponentials in normalized form:
      \begin{align*}
        i &= e^{i\frac{\pi}{2}}\\
        (1-\sqrt{3}i)&= 2\left(\frac{1}{2} - \frac{\sqrt{3}}{2}i\right) = 2e^{-i\frac{\pi}{3}}\\
        (\sqrt{3}+i)&= 2\left(\frac{\sqrt{3}}{2} + \frac{1}{2}i\right) = 2e^{i\frac{\pi}{6}}\\
      \end{align*}
      Note, $(\sqrt{3}+i)$ is $90^\circ$ rotated from $(1-\sqrt{3}i)$. This is easily verified by adding $\pi/2$ to Arg$(1-\sqrt{3}i)$.
      \begin{align*}
        \therefore i(1-\sqrt{3}i) &= e^{i\frac{\pi}{2}}2e^{-i\frac{\pi}{3}} = 2e^{i\left(\frac{\pi}{2} -\frac{\pi}{3}\right)}   \\
        2e^{i\left(\frac{3\pi}{6} -\frac{2\pi}{6}\right)} &= 2e^{\frac{\pi}{6}}  
      \end{align*}
      Including the next factor gives:
      \begin{align*}
        2e^{\frac{\pi}{6}}2e^{i\frac{\pi}{6}} &= 4e^{i\frac{2\pi}{6}} =4e^{i\frac{\pi}{3}} = \boxed{2(1+\sqrt{3}i)}
      \end{align*}
  \item[(b)] $(\sqrt{3}+i)^{6}=-64$
    
    This is the same as problem 2c with the angle being $\pi/6$ instead of the negative angle. Same work shows the result:
      \begin{align*}
        (\sqrt{3}+i)^{6} &= 2^6\left(\frac{\sqrt{3}}{2}+\frac{1}{2}i\right)^{6} \\
        2^6\left(e^{i\frac{\pi}{6}}\right)^6 &= 2^6e^{i\pi} = 2^6(-1) = \boxed{-64}\\
      \end{align*}    
    \item[(c)] $(1+\sqrt{3}i)^{-10}=2^{-11}(-1+\sqrt{3}i)$

      Here we can first rationalize the fraction
      \begin{align*}
        (1+\sqrt{3}i)^{-10} &= \left(\frac{1}{1+\sqrt{3}i}\right)^{10} = \\
        \left(\frac{1-\sqrt{3}i}{(1+\sqrt{3}i)(1-\sqrt{3}i)}\right)^{10} &= \left(\frac{1-\sqrt{3}i}{1-\sqrt{3}i+\sqrt{3}i-\sqrt{3}^2i^2}\right)^{10} =\\
        \left(\frac{1-\sqrt{3}i}{4}\right)^{10} &= 2^{-11}\left(\frac{1-\sqrt{3}i}{2}\right)^{10} = \\
        2^{-11}\left(e^{-i\frac{\pi}{3}}\right)^{10} &= 2^{-11}e^{-i\frac{10\pi}{3}} = 2^{-11}e^{-i\left(2\pi+\frac{4\pi}{3}\right)}\\
        2^{-11}e^{-i\left(\pi + \frac{1\pi}{3}\right)} &= 2^{-11}e^{-i\left(\frac{-2\pi}{3}\right)} = \boxed{2^{-11}(-1+\sqrt{3}i)}
      \end{align*}
        
\end{enumerate}

\hrule

\section*{Problem 4}
Use exponential form to find $(1-i)^{5}$

\begin{align*}
  (1-i)^{5} = \left(\frac{2}{\sqrt{2}}\right)^5\left(\frac{\sqrt{2}}{2} - \frac{\sqrt{2}}{2}i\right)^5 &= \left(\frac{2}{\sqrt{2}}\right)^5 \left(e^{-i\frac{\pi}{4}}\right)^5 =\\
  \text{angle}\ \frac{-5\pi}{4}\ \text{corresponds to}&\text{ angle}\  \frac{3\pi}{4}\\
  \left(\frac{2}{\sqrt{2}}\right)^5e^{-i\frac{5\pi}{4}} &= \left(\frac{2}{\sqrt{2}}^5\right)e^{i\frac{3\pi}{4}} = \\
  \left(\frac{2}{\sqrt{2}}\right)^5 \left(-\frac{\sqrt{2}}{2}+\frac{\sqrt{2}}{2}i\right)&=  \left(\frac{2}{\sqrt{2}}\right)^4(-1+i)\\
  \text{recall} \frac{2}{\sqrt{2}} &= \frac{2\sqrt{2}}{\sqrt{2}^2} = \sqrt{2} \\
  \therefore \left(\frac{2}{\sqrt{2}}\right)^4 &= \sqrt{2}^4 = 2^2 = 4 \\
  \therefore (1-i)^{5} &= \boxed{4(-1+i)} 
\end{align*}

\vspace{6cm}

\end{document}
%%% Local Variables:
%%% mode: latex
%%% TeX-master: t
%%% End:
