% --------------------------------------------------------------------
% LaTeX Template for Math Worksheets
% --------------------------------------------------------------------
\documentclass{article}

% --- PACKAGE IMPORTS ---
% These packages add functionality for math symbols, formatting, etc.
\usepackage[margin=.7in]{geometry}       % For setting page margins to 0.7 inch
\usepackage{amsmath, amssymb, amsthm}   % American Mathematical Society packages for advanced math
\usepackage{graphicx}                   % For including images
\usepackage{fancyhdr}                   % For creating custom headers and footers
\usepackage[colorlinks=true, urlcolor=blue, linkcolor=blue]{hyperref} % For clickable links
\usepackage{cancel}
\usepackage{array}
\usepackage{amsfonts}
\usepackage{amsxtra}
\usepackage{epsfig}
\usepackage{wasysym}
\usepackage{relsize}
\usepackage{tikz}
\tikzset{every picture/.style={scale=1.2}}
\renewcommand{\normalsize}{\fontsize{12}{20}\selectfont}

% custom commands
\newcommand{\myauthor}{Miguel Gomez}
\newcommand{\canceling}[2]{\textcolor{red}{\cancelto{\textcolor{black}{#1}}{\textcolor{black}{#2}}}}
\newcommand{\todo}[1]{\textcolor{blue}{TODO:#1}}
\newcommand{\DNE}[1]{$\therefore$ no, the limit DNE #1}
% --- DOCUMENT & AUTHOR INFORMATION ---
\title{Worksheet \# 4}
\author{
  MATH 3160 -- Complex Variables\\
  \myauthor
}
\date{Completed: \today} 

% --- HEADER & FOOTER CONFIGURATION ---
% This section sets up the header that will appear on each page.
\pagestyle{fancy}
\fancyhf{} % Clears the default header and footer
\lhead{Math 3160 -- Worksheet \# 4} % Left side of header
\rhead{\myauthor} % Puts the author's name on the right side
\rfoot{Page \thepage} % Puts the page number on the bottom right

\begin{document}

\maketitle % This command generates the title based on the information above.

% ====================================================================
% --- START OF PROBLEMS ---
% ====================================================================

% Note: \section* creates a section heading without a number.
\section*{Problem 1:}
Write the following functions $f(z)$ in the form $f(z) = u(x,y) + iv(x,y)$

\begin{enumerate}
\item[(a)] $f(z) = z^3 + z + 1$

  writing out each $z$ as $x+iy$ 
  \begin{align*}
    f(z) &= z^3 + z + 1 = (x+iy)^3 + (x+iy) + 1 \\
    (x+iy)^3 &= (x+iy)(x+iy)(x+iy) = (x^2+2ixy+i^2y^2)(x+iy) = \\
    (x^2+2ixy-y^2)(x+iy) &= x^3+ix^2y + 2ix^2y + 2i^2xy^2-y^2x-iy^3 = \\
         &= x^3- 3xy^2+i(3x^2y -y^3) \\
    f(z) &= x^3- 3xy^2+i(3x^2y -y^3) + (x+iy) + 1 = \\
    &= \boxed{(x^3- 3xy^2+ x + 1) + i(3x^2y -y^3 + y)}
  \end{align*}
\item[(b)] $f(z) = \frac{\bar{z}^2}{z}$ for $z \neq 0$
  \begin{align*}
    f(z) &= \frac{\bar{z}^2}{z} = \frac{(\overline{x+iy})^2}{(x+iy)} = \frac{(x-iy)^2}{(x+iy)} = \\
         &= \frac{(x^2-2ixy+i^2y^2)}{(x+iy)}= \frac{(x^2-2ixy-y^2)(x-iy)}{(x+iy)(x-iy)} =\\
         &= \frac{(x^3-x^2iy-2ix^3y+2i^2xy^2-y^2x+iy^3)}{(x^2+iyx-iyx -i^2y^2)} =\\
         &= \frac{(x^3-x^2iy-2ix^3y-2xy^2-y^2x+iy^3)}{(x^2+y^2)} = \\
    &= \boxed{\frac{x^3-3xy^2}{x^2+y^2} + i\frac{-3x^2y+y^3}{x^2+y^2}}
  \end{align*}
\end{enumerate}
\begin{align*}
% Your mathematical work here
\end{align*}

\vspace{1cm}
\hrule % Adds a horizontal line to separate problems.

\newpage
\section*{Problem 2:}
Consider the mapping $z \xrightarrow{} z^2$.
\begin{enumerate}
\item[(a)] What is the image of the line $z  = x + i$?
  \begin{align*}
    z \xrightarrow{} z^2  &= (x+iy) \xrightarrow{} (x+iy)^2\\
    (x^2+2ixy-y^2) &= (x^2+2ix(1)-(1)^2) = x^2 + 2xi - 1 \\
    &= (x^2-1) + i(2x)   
  \end{align*}
  
  \boxed{$\therefore$ The image of the line $x+i$ turns out to be a parabola opening to the right in $\mathbb{R}$.} Since we have $x^2$ term for $u(x,y)$ and is centered at $-1$. The parabola grows into complex plane.
  
\item[(b)] What is the image of the square bounded by the four lines $z  = \pm 1 + iy$ and $z = x \pm i$?

  case $z$ has constant real components $z  = \pm 1 + iy$:
  \begin{align*}
    z \xrightarrow{} z^2  &= (\pm 1+iy) \xrightarrow{} (\pm 1+iy)^2\\
    ((\pm 1)^2+2i(\pm 1)y-y^2) &= ((\pm 1)^2+2i(\pm 1)y-y^2) = \\
    \text{positive branch}&\\
    ((1)^2+2i(1)y-y^2) &= ((1)^2+2i(1)y-y^2) = (-y^2+1)+i(2y) \\
    \text{negative branch}&\\
    ((-1)^2+2i(-1)y-y^2) &= (1-2iy-y^2) = (-y^2+1)-i(2y)
  \end{align*}
  These two lines appear to be the same parabola that opens to the left toward $-\mathbb{R}$. if $y$ is negative, we get the same thing for $u(x,y)$ and the sign flips on $v(x,y)$. Same situation in the case $y$ is positive. 
  for the case of $x-i$:
    \begin{align*}
    z \xrightarrow{} z^2  &= (x-i) \xrightarrow{} (x-i)^2\\
    (x^2-2ix-1) &= (x^2-1) - i(2x)   
    \end{align*}
    Here, the parabola still opens to the right, and is the same parabola as we expected. with the imaginary components flipped.

    \boxed{$\therefore$ The image of the square region is the region between the two parabolas.}
\end{enumerate}

\vspace{1cm} % Space for work

\hrule

\newpage
\section*{Problem 3:}
Compute the following limits (or state that they do not exist)
\begin{enumerate}
\item[(a)] $\lim\limits_{z \to i} \frac{iz^3 - 1}{z + i}$
  \begin{align*}
    \lim\limits_{z \to i} \frac{iz^3 - 1}{z + i} &=  \frac{i(i)^3 - 1}{i + i} =\\
    \frac{i^4-1}{2i} &= \frac{1-1}{2i} = 0
  \end{align*}
\begin{center}
  \boxed{$\therefore \lim\limits_{z \to i} \frac{iz^3 - 1}{z + i} &= 0$}
\end{center}
\item[(b)] $\lim\limits_{z \to i}\left(z + \frac{1}{z}\right)$
    \begin{align*}
      \lim\limits_{z \to i}\left(z + \frac{1}{z}\right) &= \left(\lim\limits_{z \to i}z + \lim\limits_{z \to i}\frac{1}{z}\right)\\
                                                        &= \lim\limits_{z \to i}z = i \\
                                                        &= \lim\limits_{z \to i}\frac{1}{z} = \frac{1}{i} = -i
    \end{align*}
\begin{center}
  \boxed{$\therefore  \lim\limits_{z \to i}\left(z + \frac{1}{z}\right) &= i -i = 0$}
\end{center}
  \item[(c)] $\lim\limits_{z \to 0}\frac{1}{z^2}$
    
    We can evaluate this by replacing $z$ with $re^{i\theta}$ and then evaluating the limits in $r$ and $\theta$. 
    \begin{align*}
      \lim\limits_{z \to 0}\frac{1}{z^2}  &=  \lim\limits_{r \to 0}\frac{1}{(re^{i\theta})^2} = \lim\limits_{r \to 0}\frac{1}{(r)^2}e^{-i2\theta}   
    \end{align*}
    From any direction, we will end up with a div by zero issue. meaning the limit is $\infty$. We could get $\infty$ if approaching from $\theta = 0$ or we could get $-\infty$ if we approach from $\theta = \frac{\pi}{2}$

    \boxed{\DNE{because different paths give different result.}}
\end{enumerate}
\vspace{1cm} % Space for work

\hrule

% Add more problems as needed...
\newpage
\section*{Problem 4: }
Does the following limit exist?
\begin{enumerate}
\item[(a)] $\lim\limits_{z \to 0} \left(\frac{\bar{z}}{z}\right)^2$
  \begin{align*}
    \frac{\bar{z}}{z} &= \frac{\overline{re^{i\theta}}}{re^{i\theta}} = re^{-i\theta}\frac{1}{r}e^{-i\theta} = e^{-i2\theta}\\
                       \text{squaring this then doubles }&\text{the angle theta and we see there is no more dependence on r}\\
                      &=\lim\limits_{z \to 0} e^{-i4\theta}
  \end{align*}
  Approaching from $\theta = 0$ we get $1$, but approaching from $\theta = \frac{\pi}{4}$ we get $-1$. 

 \boxed{ \DNE{because different paths give different result.}}
\end{enumerate}
\hrule

\end{document}

%%% Local Variables:
%%% mode: latex
%%% TeX-master: t
%%% End:
