% --------------------------------------------------------------------
% LaTeX Template for Math Worksheets
% --------------------------------------------------------------------

\documentclass{article}

% --- PACKAGE IMPORTS ---
% These packages add functionality for math symbols, formatting, etc.
\usepackage[margin=.7in]{geometry}       % For setting page margins to 1 inch
\usepackage{amsmath, amssymb, amsthm}   % American Mathematical Society packages for advanced math
\usepackage{graphicx}                   % For including images
\usepackage{fancyhdr}                   % For creating custom headers and footers
\usepackage[colorlinks=true, urlcolor=blue, linkcolor=blue]{hyperref} % For clickable links
\usepackage{cancel}


% custom commands
\newcommand{\myauthor}{Miguel Gomez}
\newcommand{\canceling}[2]{\textcolor{red}{\cancelto{\textcolor{black}{#1}}{\textcolor{black}{#2}}}}
\newcommand{\todo}[1]{\textcolor{blue}{TODO:#1}}
% --- DOCUMENT & AUTHOR INFORMATION ---
\title{Worksheet 1: Complex Numbers}
\author{
  MATH 3160\\
  \myauthor
}
\date{Completed: August 19, 2025} 

% --- HEADER & FOOTER CONFIGURATION ---
% This section sets up the header that will appear on each page.
\pagestyle{fancy}
\fancyhf{} % Clears the default header and footer
\lhead{Math 3160 -- Worksheet 1} % Centered text in the header
\rhead{\myauthor} % Puts the author's name (defined above) on the right side
\rfoot{Page \thepage} % Puts the page number on the bottom right

\begin{document}

\maketitle % This command generates the title based on the information above.

% ====================================================================
% --- START OF PROBLEMS ---
% ====================================================================

% Note: \section* creates a section heading without a number.
\section*{Problem 1: Simplify}
Simplify the following complex expressions. 

% The 'enumerate' environment creates a numbered list.
% Using \item[(a)] allows for custom labels like (a), (b), etc.
\begin{enumerate}
    \item[(a)] $(\sqrt{2}-i)-i(1-\sqrt{2}i)$ 
    
      \vspace{.1cm} % Adds vertical space for handwritten work or can be removed for typed solutions.
      \begin{align*}
         (\sqrt{2}-i)-&i(1-\sqrt{2}i)=\\
         (\sqrt{2}-i)-&i-\sqrt{2}i^2=\\
         (\sqrt{2}-i)-&i-\sqrt{2}\canceling{-1}{i^2}=\\
         \sqrt{2}-i-&i+\sqrt{2}=\\
         \sqrt{2}+\sqrt{2}-&i-i=\\
         2\sqrt{2}-&2i = 2(\sqrt{2}-i)  
       \end{align*}
    
    \item[(b)] $(2-3i)(-2+i)$ 
    \begin{align*}
      (2-3i)&(-2+i)=\\
      (-2(2-3i)+&i(2-3i))=\\
      (-4+6i)+&(2i-3i^2)=\\
      (-4+6i)+&(2i-3\canceling{-1}{i^2})=\\
      (-4+6i)+&(2i+3)=\\
      (-4 + 3)+&(2i+6i)=(-1 + 8i)
    \end{align*}
    
    \item[(c)] $(3+i)(3-i)\left(\frac{1}{5}+\frac{1}{10}i\right)$ 
    
   \begin{align*}
     (3+i)(3-i)&\left(\frac{1}{5}+\frac{1}{10}i\right)=\\
     (3(3+i)-i(3+i))&\left(\frac{1}{5}+\frac{1}{10}i\right)=\\
     ((9+3i)-3i-i^2))&\left(\frac{1}{5}+\frac{1}{10}i\right)=\\
     ((9+3i)-3i-\canceling{-1}{i^2}))&\left(\frac{1}{5}+\frac{1}{10}i\right)=\\
     (9+\canceling{0}{3i-3i}+1))&\left(\frac{1}{5}+\frac{1}{10}i\right)=\\
     10&\left(\frac{1}{5}+\frac{1}{10}i\right) = (2 + i)\\
   \end{align*}
    
\end{enumerate}

\hrule % Adds a horizontal line to separate problems.
\newpage
\section*{Problem 2: Verification}
Verify that each of the two numbers $z=1\pm i$ satisfies the equation $z^{2}-2z+2=0$.

\hrule
\vspace{.25cm}
First root: $1 + i$
\begin{align*}
  z^{2}-2z+2&=0\\
  (1+i)^{2}-2(1+i)+2&=0\\
  (1+i)^{2}-2(1+i)+2&=0\\
  (1 + 2i + i^2) - 2 - 2i + 2&=0\\
  (1 + \canceling{-1}{i^2}) + (2i - 2i)  + (2 - 2)&=0\\
  \canceling{0}{(1 - 1)} + \canceling{0}{(2i - 2i)}  + \canceling{0}{(2 - 2)} &= 0\\
  0 &= 0 \qed
\end{align*}
\vspace{.25cm}
\hrule
\vspace{.25cm}
Second root: $1 - i$
\begin{align*}
  z^{2}-2z+2&=0\\
  (1-i)^{2}-2(1-i)+2&=0\\
  (1-i)^{2}-2(1-i)+2&=0\\ 
  (1-2i +i^2)-21+2i+2&=0\\
  (1-2i +\canceling{-1}{i^2})-2+2i+2&=0\\
  \canceling{0}{(1-1)} + \canceling{0}{(2i-2i)}+\canceling{0}{(2-2)}&=0\\
  0 &= 0 \qed
\end{align*}
\newpage
\section*{Problem 3: Solving Equations}
Solve the equation $z^{2}+z+1=0$ for $z=(x,y)$ by writing $(x,y)(x,y)+(x,y)+(1,0)=(0,0)$ and then solving a pair of simultaneous equations in $x$ and $y$. 
\vspace{.25cm}
\hrule
\vspace{.25cm}
Using expressions (3) and (4) from the textbook shown below:

\begin{align*}
  (3)\ (x_1,y_1) + (x_2,y_2) &= (x_1 + x_2,  y_1+y_2)\\
  (4)\ (x_1,y_1)(x_2,y_2) &= (x_1x_2 - y_1y_2, y_1x_2 + x_1y_2)
\end{align*}

We can see that $x_1 = x_2$ and $y_1 = y_2$ for our problem. Solving for the expression in the form above:
\begin{align*}
(x,y)(x,y)+(x,y)+(1,0)&=(0,0)\\
(x^2 - y^2 + x + 1, 2xy + y + 0) &=(0,0)
\end{align*}
Case $x = 0$:
\begin{align*}
(x^2 - y^2 + x + 1, 2xy + y + 0) &=(0,0)\\
(\canceling{0}{x^2} - y^2 + \canceling{0}{x} + 1, \canceling{0}{2xy} + y + 0) &=(0,0)\\  
( -y^2 +  1,  y + 0) &=(0,0) 
\end{align*}
We can see that we simultaneously have expressions $y=0$ and $y = \pm1$. This is a contradiction.\\
\noindent

Case $y = 0$:
\begin{align*}
(x^2 - y^2 + x + 1, 2xy + y + 0) &=(0,0)\\
(x^2 - \canceling{0}{y^2} + x + 1, \canceling{0}{2xy} + \canceling{0}{y} + 0) &=(0,0)\\
(x^2 + x + 1,   0) &=(0,0)
\end{align*}
This is not factorable in $\mathbb{R}$. We can show this with the quadratic equation:
\begin{align*}
  x &= \frac{-b \pm \sqrt{b^2-4ac}}{2a}\\
  x &= \frac{-1 \pm \sqrt{1^2-4(1)(1)}}{2(1)}\\
  x &= \frac{-1 \pm \sqrt{-3}}{2}\ \ ; i = \sqrt{-1}\\
  x &= \frac{-1 \pm i\sqrt{3}}{2} = -\frac{1}{2} \pm i\frac{\sqrt{3}}{2} 
\end{align*}
$\therefore$ if $x$ is complex, that implies that $y$ is non-zero. Then we can plug $x = -\frac{1}{2}$ and $y = \pm\frac{\sqrt{3}}{2}$ back into the starting expression $(x^2 - y^2 + x + 1, 2xy + y + 0)$ to verify the answers come out to $(0,0)$.

\begin{align*}
  (x^2 - y^2 + x + 1\ , 2xy + y + 0) &= (0,\ 0)\\
  \left(  \left(-\frac{1}{2}\right)^2 - \left(\pm\frac{\sqrt{3}}{2}\right)^2 + \left(-\frac{1}{2}\right) + 1,\  2\left(-\frac{1}{2}\right)\left(\pm\frac{\sqrt{3}}{2}\right) + \left(\pm\frac{\sqrt{3}}{2}\right) + 0 \right) &= (0,0)\\
  \left(  \frac{1}{4} - \frac{3}{4} -  \frac{1}{2} + 1,\  \canceling{-1}{\left(-\frac{2}{2}\right)}\left(\pm\frac{\sqrt{3}}{2}\right) + \left(\pm\frac{\sqrt{3}}{2}\right) \right) &= (0,0)\\
  \left(  \canceling{-\frac{1}{2}}{\frac{1}{4} - \frac{3}{4}} -  \frac{1}{2} + 1,\  \left(\canceling{\mp}{-1 \pm}\ \ \ \frac{\sqrt{3}}{2}\right) + \left(\pm\frac{\sqrt{3}}{2}\right) \right) &= (0,0)\\
  \left( \canceling{-1}{-\frac{1}{2} -  \frac{1}{2}} + 1,\  \left(\mp\frac{\sqrt{3}}{2}\right) + \left(\pm\frac{\sqrt{3}}{2}\right) \right) &= (0,0)\\  
  \left( \canceling{0}{1 - 1},\  \canceling{0}{\left(\mp\frac{\sqrt{3}}{2}\right) + \left(\pm\frac{\sqrt{3}}{2}\right)} \right) &= (0,0)\\
  (0, 0)&=(0, 0) \qed
\end{align*}
\end{document}


%%% Local Variables:
%%% mode: latex
%%% TeX-master: t
%%% End:
