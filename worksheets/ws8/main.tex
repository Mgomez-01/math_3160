% --------------------------------------------------------------------
% LaTeX Template for Math Worksheets
% --------------------------------------------------------------------
\documentclass{article}

% --- PACKAGE IMPORTS ---
% These packages add functionality for math symbols, formatting, etc.
\usepackage[margin=.7in]{geometry}       % For setting page margins to 0.7 inch
\usepackage{amsmath, amssymb, amsthm}   % American Mathematical Society packages for advanced math
\usepackage{graphicx}                   % For including images
\usepackage{fancyhdr}                   % For creating custom headers and footers
\usepackage[colorlinks=true, urlcolor=blue, linkcolor=blue]{hyperref} % For clickable links
\usepackage{cancel}
\usepackage{array}
\usepackage{amsfonts}
\usepackage{amsxtra}
\usepackage{epsfig}
\usepackage{wasysym}
\usepackage{relsize}
\usepackage{tikz}
\tikzset{every picture/.style={scale=1.2}}
\usetikzlibrary{decorations.markings} 
\renewcommand{\normalsize}{\fontsize{12}{20}\selectfont}

% custom commands
\newcommand{\myauthor}{Miguel Gomez}
\newcommand{\canceling}[2]{\textcolor{red}{\cancelto{\textcolor{black}{#1}}{\textcolor{black}{#2}}}}
\newcommand{\todo}[1]{\textcolor{blue}{TODO:#1}}
% Save the original commands
\let\oldcos\cos
\let\oldsin\sin
\let\oldcosh\cosh
\let\oldsinh\sinh

% Redefine with automatic parentheses
\renewcommand{\cos}[1]{\oldcos\left(#1\right)}
\renewcommand{\sin}[1]{\oldsin\left(#1\right)}
\renewcommand{\cosh}[1]{\oldcosh\left(#1\right)}
\renewcommand{\sinh}[1]{\oldsinh\left(#1\right)}

\newcommand{\der}[2]{\frac{d#1}{d#2}}
\newcommand{\secder}[2]{\frac{d^2#1}{d#2^2}}
\newcommand{\parder}[2]{\frac{\partial#1}{\partial#2}}
\newcommand{\secparder}[2]{\frac{\partial^2#1}{\partial#2^2}}

% --- DOCUMENT & AUTHOR INFORMATION ---
\title{Worksheet \# 8}
\author{
  MATH 3160 -- Complex Variables\\
  \myauthor
}
\date{Completed: \today} 

% --- HEADER & FOOTER CONFIGURATION ---
% This section sets up the header that will appear on each page.
\pagestyle{fancy}
\fancyhf{} % Clears the default header and footer
\lhead{Math 3160 -- Worksheet \# 8} % Left side of header
\rhead{\myauthor} % Puts the author's name on the right side
\rfoot{Page \thepage} % Puts the page number on the bottom right

\begin{document}

\maketitle % This command generates the title based on the information above.

% ====================================================================
% --- START OF PROBLEMS ---
% ====================================================================

% Note: \section* creates a section heading without a number.
\section*{Problem 1}
A contour $C$ is parametrized by $\gamma(t) = e^{i(\pi - t)} \quad (0 \leq t \leq \pi)$. Draw the contour $C$, carefully indicating its starting point and ending point. 

\begin{align*}
% Your mathematical work here
\end{align*}

\vspace{1cm} % Add space for additional work if needed
% For graphs/diagrams, you can use TikZ:
\begin{center}
\begin{tikzpicture}
    % Your TikZ code here
    % Example: Draw axes
    \draw[->] (-2,0) -- (3,0) node[right] {$\mathbb{R}$};
    \draw[->] (0,-2) -- (0,3) node[above] {$\mathbb{I}$};
\end{tikzpicture}
\end{center}
\vspace{1cm}
\hrule % Adds a horizontal line to separate problems.

\newpage
\section*{Problem 2}
Write down the parametrization of the following contour:
\begin{center}
\begin{tikzpicture}
    % Your TikZ code here
    % Example: Draw axes
    \draw[->] (-2,0) -- (3,0) node[right] {};
    \draw[->] (0,-2) -- (0,3) node[above] {};
            \draw[->, decoration={markings, 
        mark=at position 0.25 with {\arrow{>}},
        mark=at position 0.5 with {\arrow{>}},
        mark=at position 0.75 with {\arrow{>}}},
        postaction={decorate}, thick]  (2, 0) -- (2,2) node[right] {$2+2i$};
    \draw[->] (2, -.15) -- (2,.15) node[below=3mm] {$2$};
    \draw[->] (-2, -.15) -- (-2,.15) node[below=3mm] {$-2$};
                \draw[<-, decoration={markings, 
        mark=at position 0.25 with {\arrow{<}},
        mark=at position 0.5 with {\arrow{<}},
        mark=at position 0.75 with {\arrow{<}}},
      postaction={decorate}, thick] (2,0) arc (0:180:2);
      \draw (1,1.7) node[above] {$\gamma_1$};
      \draw (2,1) node[right] {$\gamma_2$};
      \draw[thick, blue, fill=blue!100] (-2,0) circle (0.05);
      \draw[thick, red, fill=red!100] (2,0) circle (0.05);
      \draw[thick, purple, fill=purple!100] (2,2) circle (0.05);
    
\end{tikzpicture}
\end{center}
\vspace{0.5cm}
\hrule % Adds a horizontal line to separate problems.
\vspace{0.5cm} % Space for work
Starting from the blue point above, we move in a circular path along the arc, landing us at the red point. The following is the parametrization of that arc:
\begin{align*}
  \gamma_1(t) : \ [0,1] \xrightarrow{} 2e^{-i\pi (t + 1)} =2e^{-i(\pi t) }e^{ -i\pi} \quad 0 \leq t \leq 1 
\end{align*}
We start gamma at $\pi$ by including the factor of $e^{-i\pi}$. Then as $t$ sweeps from 0 to 1, we end at $e^{-i2\pi}$, effectively rotating the semicircular path on the bottom of the circle around the origin. Then for $\gamma_2$ we will then do the following:
\begin{align*}
  \gamma_2 : [1,2] \xrightarrow{} 2+2i(t-1) \quad 1 \leq t \leq 2
\end{align*}

\vspace{0.5cm} % Space for work

\hrule

\newpage
\section*{Problem 3}


\vspace{1cm} % Space for work

\hrule

% Add more problems as needed...
% \newpage
% \section*{Problem 4}
% 
% \hrule

\end{document}
