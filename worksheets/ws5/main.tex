% --------------------------------------------------------------------
% LaTeX Template for Math Worksheets
% --------------------------------------------------------------------
\documentclass{article}

% --- PACKAGE IMPORTS ---
% These packages add functionality for math symbols, formatting, etc.
\usepackage[margin=.6in]{geometry}       % For setting page margins to 0.7 inch
\usepackage{amsmath, amssymb, amsthm}   % American Mathematical Society packages for advanced math
\usepackage{graphicx}                   % For including images
\usepackage{fancyhdr}                   % For creating custom headers and footers
\usepackage[colorlinks=true, urlcolor=blue, linkcolor=blue]{hyperref} % For clickable links
\usepackage{cancel}
\usepackage{array}
\usepackage{amsfonts}
\usepackage{amsxtra}
\usepackage{epsfig}
\usepackage{wasysym}
\usepackage{relsize}
\usepackage{tikz}
\tikzset{every picture/.style={scale=1.2}}
\renewcommand{\normalsize}{\fontsize{12}{20}\selectfont}

% custom commands
\newcommand{\myauthor}{Miguel Gomez}
\newcommand{\canceling}[2]{\textcolor{red}{\cancelto{\textcolor{black}{#1}}{\textcolor{black}{#2}}}}
\newcommand{\todo}[1]{\textcolor{blue}{TODO:#1}}
\newcommand{\der}[2]{\frac{d#1}{d#2}}
\newcommand{\part}[2]{\frac{\partial#1}{\partial#2}}
% --- DOCUMENT & AUTHOR INFORMATION ---
\title{Worksheet \# 4}
\author{
	MATH 3160 -- Complex Variables\\
	\myauthor
}
\date{Completed: \today}

% --- HEADER & FOOTER CONFIGURATION ---
% This section sets up the header that will appear on each page.
\pagestyle{fancy}
\fancyhf{} % Clears the default header and footer
\lhead{Math 3160 -- Worksheet \# 4} % Left side of header
\rhead{\myauthor} % Puts the author's name on the right side
\rfoot{Page \thepage} % Puts the page number on the bottom right

\begin{document}

\maketitle % This command generates the title based on the information above.

% ====================================================================
% --- START OF PROBLEMS ---
% ====================================================================

% Note: \section* creates a section heading without a number.
\section*{Problem 1}
Recall that we have defined the complex exponential function $e^z$ by the formula $e^z = e^xe^{iy} = e^x(\cos{(y)} + i \sin{(y)})$, where $x = Re(z)$ and $y = Im(z)$.
\\

Calculate $f'(z)$ for each of the following functions:

\begin{enumerate}
	\item[(a)] $f (z) = (z + 2)^5$
		\begin{align*}
			f (z)    & = (z + 2)^5                           \\
			f'(z)    & = \der{f}{z}                          \\
			f'(g(z)) & = \der{f}{g}\der{g}{z} =  5(z+2)^4(1)
		\end{align*}
	\item[(b)] $f (z) = e^{z^3+z+1}$
		\begin{align*}
			f(z)             & = e^{z^3+z+1} = e^{z^3}e^{z}e^{1} = e(e^{z^3}e^{z})  \\
			\der{(ab)}{z}    & = a'b+ab'                                          \\
			a                & = e^{z^3} \to a' = e^{z^3}(3z^2)                    \\
			b                & = e^{z} \to b' = e^z(1)                             \\
			a'b+ab'          & = e^{z^3}(3z^2)e^z+e^{z^3}e^z = e^{z^3}e^z(3z^2 + 1) \\
			\therefore f'(z) & = e^{z^3}e^ze(3z^2 + 1) = e^{z^3+z+1}(3z^2 + 1)
		\end{align*}
	\item[(c)] $f (z) = e^{1/z}$
		\begin{align*}
			f'(z) & = \der{}{z}e^{1/z} = \der{}{z}e^{z^{-1}} = e^{z^{-1}}(-1)z^{-2} \\
			      & = -e^{1/z}\frac{1}{z^2}
		\end{align*}
\end{enumerate}

\vspace{1cm}
\hrule % Adds a horizontal line to separate problems.

\section*{Problem 2}
Use the Cauchy-Riemann equations to show that $f'(z)$ does not exist at any point for the following:

\begin{enumerate}
	\item[(a)] $f (z) = z - \bar z$
	\item[(b)] $f (z) = e^xe^{-iy}$
\end{enumerate}
The Cauchy-Riemann equations are the following:
\begin{align*}
	\der{u}{x} & = \der{v}{y}  \\
	\der{u}{y} & = -\der{v}{x}
\end{align*}
$f'(z_0)  \iff$ these expressions above hold when evaluated at $(x_0,y_0)$.

\subsection*{(a)}
\begin{align*}
	f (z) = z - \bar z & = (x+iy) - (x-iy) = (x-x) + i(y+y) = i2y                                           \\
	u(x,y)             & = 0 \quad \quad v(x,y) = 2y                                                        \\
	\der{u}{x}         & = 0 \quad \quad \der{v}{y} = 2 \quad\quad \der{u}{y} = 0 \quad\quad \der{v}{x} = 0 \\
	\der{u}{x}         & \neq \der{v}{y}                                                                    \\
	\therefore f'(z)\ DNE
\end{align*}
\subsection*{(b)}
\begin{align*}
	f (z)         & = e^xe^{-iy} = e^x(\cos{(-y)} +i\sin{(-y)}) = e^x(\cos{(y)} - i\sin{(y)})                                                         \\
	u(x,y)        & = e^x\cos{(y)} \quad \quad v(x,y) = -e^x\sin{(y)}                                                                                 \\
	\der{u}{x}    & = e^x\cos{(y)} \quad \quad \der{v}{y} = -e^x\cos{(y)} \quad\quad \der{u}{y} = -e^x\sin{(y)} \quad\quad \der{v}{x} = -e^x\sin{(y)} \\
	\der{u}{x}    & =  e^x\cos{(y)}                                                                                                                   \\
	\der{v}{y}    & = -e^x\cos{(y)}                                                                                                                   \\
	e^x\cos{(y)}  & = -e^x\cos{(y)} \to 2e^x\cos{(y)} = 0                                                                                             \\
	              & \text{Only possible if }\cos{(y)} = 0\ \text{ and similar situation for the other equation}                                       \\
	\der{u}{y}    & = -\der{v}{x}                                                                                                                     \\
	-e^x\sin{(y)} & = e^x\sin{(y)} \to 2e^x\sin{(y)} = 0                                                                                              \\
	              & \sin{(\theta)}\text{ and }\cos{(\theta)} \text{ cannot both be $0$ simultaneously}                                                \\
	              & \therefore f'(z)\ DNE
\end{align*}

\end{document}

%%% Local Variables:
%%% mode: latex
%%% TeX-master: t
%%% End: