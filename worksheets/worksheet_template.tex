% --------------------------------------------------------------------
% LaTeX Template for Math Worksheets
% --------------------------------------------------------------------
\documentclass{article}

% --- PACKAGE IMPORTS ---
% These packages add functionality for math symbols, formatting, etc.
\usepackage[margin=.7in]{geometry}       % For setting page margins to 0.7 inch
\usepackage{amsmath, amssymb, amsthm}   % American Mathematical Society packages for advanced math
\usepackage{graphicx}                   % For including images
\usepackage{fancyhdr}                   % For creating custom headers and footers
\usepackage[colorlinks=true, urlcolor=blue, linkcolor=blue]{hyperref} % For clickable links
\usepackage{cancel}
\usepackage{array}
\usepackage{amsfonts}
\usepackage{amsxtra}
\usepackage{epsfig}
\usepackage{wasysym}
\usepackage{relsize}
\usepackage{tikz}
\tikzset{every picture/.style={scale=1.2}}
\renewcommand{\normalsize}{\fontsize{12}{20}\selectfont}

% custom commands
\newcommand{\myauthor}{Miguel Gomez}
\newcommand{\canceling}[2]{\textcolor{red}{\cancelto{\textcolor{black}{#1}}{\textcolor{black}{#2}}}}
\newcommand{\todo}[1]{\textcolor{blue}{TODO:#1}}

% --- DOCUMENT & AUTHOR INFORMATION ---
\title{Worksheet \# x}
\author{
  MATH 3160 -- Complex Variables\\
  \myauthor
}
\date{Completed: \today} 

% --- HEADER & FOOTER CONFIGURATION ---
% This section sets up the header that will appear on each page.
\pagestyle{fancy}
\fancyhf{} % Clears the default header and footer
\lhead{Math 3160 -- Worksheet \# x} % Left side of header
\rhead{\myauthor} % Puts the author's name on the right side
\rfoot{Page \thepage} % Puts the page number on the bottom right

\begin{document}

\maketitle % This command generates the title based on the information above.

% ====================================================================
% --- START OF PROBLEMS ---
% ====================================================================

% Note: \section* creates a section heading without a number.
\section*{Problem 1}


\begin{align*}
% Your mathematical work here
\end{align*}

\vspace{1cm} % Add space for additional work if needed
% For graphs/diagrams, you can use TikZ:
\begin{center}
\begin{tikzpicture}
    % Your TikZ code here
    % Example: Draw axes
    \draw[->] (-2,0) -- (3,0) node[right] {$\mathbb{R}$};
    \draw[->] (0,-2) -- (0,3) node[above] {$\mathbb{I}$};
\end{tikzpicture}
\end{center}
\vspace{1cm}
\hrule % Adds a horizontal line to separate problems.

\newpage
\section*{Problem 2}


\vspace{1cm} % Space for work

\hrule

\newpage
\section*{Problem 3}


\vspace{1cm} % Space for work

\hrule

% Add more problems as needed...
% \newpage
% \section*{Problem 4}
% 
% \hrule

\end{document}
