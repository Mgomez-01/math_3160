% --------------------------------------------------------------------
% LaTeX Template for Math Homework
% --------------------------------------------------------------------

\documentclass{article}

% --- PACKAGE IMPORTS ---
% These packages add functionality for math symbols, formatting, etc.
\usepackage[margin=.7in]{geometry}       % For setting page margins
\usepackage{amsmath, amssymb, amsthm}   % American Mathematical Society packages for advanced math
\usepackage{graphicx}                   % For including images
\usepackage{fancyhdr}                   % For creating custom headers and footers
\usepackage[colorlinks=true, urlcolor=blue, linkcolor=blue]{hyperref} % For clickable links
\usepackage{cancel}
\usepackage{array}
\usepackage{amsfonts}
\usepackage{amsxtra}
\usepackage{epsfig}
\usepackage{wasysym}
\usepackage{relsize}
\usepackage{tikz}
\tikzset{every picture/.style={scale=1.2}}
\renewcommand{\normalsize}{\fontsize{12}{20}\selectfont}

% custom commands
\newcommand{\myauthor}{Miguel Gomez}
\newcommand{\canceling}[2]{\textcolor{red}{\cancelto{\textcolor{black}{#1}}{\textcolor{black}{#2}}}}
\newcommand{\todo}[1]{\textcolor{blue}{TODO:#1}}
% Save the original commands
\let\oldcos\cos
\let\oldsin\sin
\let\oldcosh\cosh
\let\oldsinh\sinh

% Redefine with automatic parentheses
\renewcommand{\cos}[1]{\oldcos\left(#1\right)}
\renewcommand{\sin}[1]{\oldsin\left(#1\right)}
\renewcommand{\cosh}[1]{\oldcosh\left(#1\right)}
\renewcommand{\sinh}[1]{\oldsinh\left(#1\right)}

\newcommand{\der}[2]{\frac{d#1}{d#2}}
\newcommand{\secder}[2]{\frac{d^2#1}{d#2^2}}
\newcommand{\parder}[2]{\frac{\partial#1}{\partial#2}}
\newcommand{\secparder}[2]{\frac{\partial^2#1}{\partial#2^2}}

% --- DOCUMENT & AUTHOR INFORMATION ---
\title{Homework \# 8}
\author{
  MATH 3160 -- Complex Variables\\
  \myauthor
}
\date{Completed: \today}

% --- HEADER & FOOTER CONFIGURATION ---
% This section sets up the header that will appear on each page.
\pagestyle{fancy}
\fancyhf{} % Clears the default header and footer
\lhead{Math 3160 -- HW \# 8} % Left side of header
\rhead{\myauthor} % Puts the author's name on the right side
\rfoot{Page \thepage} % Puts the page number on the bottom right

\begin{document}

\maketitle % This command generates the title based on the information above.

% ====================================================================
% --- START OF PROBLEMS ---
% ====================================================================

\section*{Problem 1}
Find the Taylor series expansion at $z_0=0$ of the function \[ f(z) = \frac{z}{z^4+9} = \frac{z}{9} \cdot \frac{1}{1+\big( \frac{z^4}{9}\big)} \]

\vspace{.5cm} % Space between problems

\hrule % Adds a horizontal line to separate problems.
\newpage
\section*{Problem 2}
Find the Taylor series expansion of \[ f(z) = \frac{1}{1-z} \] at the following points, if it exists (or say it doesn't exist if the series diverges).

\begin{enumerate}
 \item[(a)] $z=0$
 \item[(b)] $z=1$
 \item[(c)] $z=2$
\end{enumerate}


\vspace{.5cm} % Space between problems

\hrule

\newpage
\section*{Problem 3}
Derive the Laurent series representation of  \[ f(z) = \frac{e^z}{(z+1)^2} \]  about the point $z_0=-1$.

\vspace{.5cm} % Space for work

\hrule

\newpage
\section*{Problem 4}
Represent the function 	\[ f(z) = \frac{z+1}{z-1} \]
\begin{enumerate}
\item[(a)] By a Taylor series at $z_0=0$ and state its domain of convergence.
\item[(b)] By a Laurent series about $z_0 = 1$.
\end{enumerate}


\vspace{.5cm} % Space for work

\hrule


\end{document}

%%% Local Variables:
%%% mode: latex
%%% TeX-master: t
%%% End:
