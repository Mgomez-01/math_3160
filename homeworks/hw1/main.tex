% --------------------------------------------------------------------
% LaTeX Template for Math Homework
% --------------------------------------------------------------------

\documentclass{article}

% --- PACKAGE IMPORTS ---
% These packages add functionality for math symbols, formatting, etc.
\usepackage[margin=.7in]{geometry}       % For setting page margins
\usepackage{amsmath, amssymb, amsthm}   % American Mathematical Society packages for advanced math
\usepackage{graphicx}                   % For including images
\usepackage{fancyhdr}                   % For creating custom headers and footers
\usepackage[colorlinks=true, urlcolor=blue, linkcolor=blue]{hyperref} % For clickable links
\usepackage{cancel}
\usepackage{array}
\usepackage{amsfonts}
\usepackage{amsxtra}
\usepackage{epsfig}
\usepackage{wasysym}
\usepackage{relsize}
\renewcommand{\normalsize}{\fontsize{12}{20}\selectfont}



% custom commands
\newcommand{\myauthor}{Miguel Gomez}
\newcommand{\canceling}[2]{\textcolor{red}{\cancelto{\textcolor{black}{#1}}{\textcolor{black}{#2}}}}
\newcommand{\todo}[1]{\textcolor{blue}{TODO:#1}}

% --- DOCUMENT & AUTHOR INFORMATION ---
\title{Homework 1: Complex Numbers}
\author{
  MATH 3160\\
  \myauthor
}
\date{Completed: \today}

% --- HEADER & FOOTER CONFIGURATION ---
% This section sets up the header that will appear on each page.
\pagestyle{fancy}
\fancyhf{} % Clears the default header and footer
\lhead{Math 3160 -- HW 1} % Left side of header
\rhead{\myauthor} % Puts the author's name on the right side
\rfoot{Page \thepage} % Puts the page number on the bottom right

\begin{document}

\maketitle % This command generates the title based on the information above.

% ====================================================================
% --- START OF PROBLEMS ---
% ====================================================================

\section*{Problem 1: Complex Number Reduction}
Reduce each of these to a real number:
\begin{enumerate}
\item[(a)] $ \frac{1+2i}{3-4i} + \frac{2-i}{5i}$
  \begin{align*}
    \frac{1+2i}{3-4i} &+ \frac{2-i}{5i} = \\
    \frac{(1+2i)(3+4i)}{(3-4i)(3+4i)} &+ \frac{(2-i)(-5i)}{(5i)(-5i)} = \\
    \frac{(1+2i)(3+4i)}{9-16i^2} &+ \frac{(2-i)(-5i)}{-25i^2} = \\
    \frac{(3+4i+6i+8\canceling{-1}{i^2})}{9-16\canceling{-1}{i^2}} &+ \frac{(-10i+5\canceling{-1}{i^2})}{-25\canceling{-1}{i^2}} = \\
    \frac{(3+4i+6i-8)}{25} &+ \frac{(-10i-5)}{25} = \\
    \frac{(3+4i+6i-8)}{25} &+ \frac{(-10i-5)}{25} = \\
    \frac{(-5+10i)}{25} &+ \frac{(-10i-5)}{25} = \\
    \frac{(-5-5+10i -10i)}{25} = -\frac{2}{5}\\
  \end{align*}
\item[(b)] $ \frac{5i}{(1-i)(2-i)(3-i)}$
  \begin{align*}
    \frac{5i}{(1-i)(2-i)(3-i)} &= \frac{5i}{(2-i-2i-i^2)(3-i)} =\\
    \frac{5i}{(2-3i-\canceling{-1}{i^2})(3-i)} &= \frac{5i}{(2-3i+1)(3-i)} =\\
    \frac{5i}{(3-3i)(3-i)} &= \frac{5i}{(9-3i-9i-3i^2)} =\\
    \frac{5i}{(9-12i-3\canceling{-1}{i^2})}\ \ &= \frac{5i}{(12-12i)} =\\
    \frac{5i}{(12-12i)}\cdot\frac{(12+12i)}{(12+12i)} &= \frac{5i}{(12-12i)}\cdot\frac{(12+12i)}{(12+12i)}
  \end{align*}
\item[(c)] $ (1-i)^4$
\end{enumerate}

\vspace{.5cm} % Space for work

\hrule % Adds a horizontal line to separate problems.

\newpage
\section*{Problem 2: Vector Addition and Subtraction}
Locate the numbers $z_1+z_2$ and $z_1 - z_2$ vectorially by drawing a graph when:

\begin{enumerate}
    \item[(a)] $z_1 = 2i$, $z_2 = 2/3-i$
    
    \vspace{.5cm} % Space for graph
    
    \item[(b)] $z_1 = -\sqrt{3}+i$, $z_2 = \sqrt{3}$
    
    \vspace{.5cm} % Space for graph
    
    \item[(c)] $z_1 = (3,1)$, $z_2 = (1,4)$
    
    \vspace{.5cm} % Space for graph
    
    \item[(d)] $z_1 = x_1+iy_1$, $z_2 = x_1-iy_1$
    
    \vspace{.5cm} % Space for graph
\end{enumerate}

\hrule

\newpage
\section*{Problem 3: Geometric Sets in the Complex Plane}
In each case, sketch the set of points determined by the given condition:

\begin{enumerate}
    \item[(a)] $|z-1+i|=1$
    
    \vspace{.5cm} % Space for sketch
    
    \item[(b)] $|z+i| \leq 3$
    
    \vspace{.5cm} % Space for sketch
    
    \item[(c)] $|z-4i|\geq4$
    
    \vspace{.5cm} % Space for sketch
\end{enumerate}

\textit{Hint: Note that for any two complex numbers $z_1, z_2$, the absolute value $|z_1 - z_2|$ is the distance between $z_1$ and $z_2$ in the complex plane.}

\hrule

\newpage
\section*{Problem 4: Principal Arguments}
Find the principal argument $\text{Arg}(z)$ when:

\begin{enumerate}
    \item[(a)] $z = \frac{i}{-2-2i}$
    
    \vspace{.5cm} % Space for work
    
    \item[(b)] $z = (\sqrt{3}-i)^6$
    
    \vspace{.5cm} % Space for work
\end{enumerate}

Show your work.

\hrule

\newpage
\section*{Problem 5: Argument Properties}
For any two non-zero complex numbers $z_1, z_2$, show that any angle $\theta$ in the set $\arg{(z_1z_2)}$ can be written as
\[ \theta = \theta_1 + \theta_2 \]
where $\theta_1 \in \arg(z_1)$ and $\theta_2 \in \arg(z_2)$. Also find an example where the principal argument $\text{Arg} (z_1z_2)$ is not equal to $\text{Arg} (z_1) + \text{Arg} (z_2)$.

\vspace{.5cm} % Space for proof and example

\hrule

\newpage
\section*{Problem 6: Principal Argument Addition}
Show that if $\text{Re}(z_1)>0$ and $\text{Re}(z_2)>0$, then $\text{Arg}(z_1z_2) = \text{Arg}(z_1)+\text{Arg}(z_2)$. Use polar form of $z_1$ and $z_2$ to do so.

\vspace{.5cm} % Space for proof

\end{document}