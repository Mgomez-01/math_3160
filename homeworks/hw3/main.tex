% --------------------------------------------------------------------
% LaTeX Template for Math Homework
% --------------------------------------------------------------------

\documentclass{article}

% --- PACKAGE IMPORTS ---
% These packages add functionality for math symbols, formatting, etc.
\usepackage[margin=.7in]{geometry}       % For setting page margins
\usepackage{amsmath, amssymb, amsthm}   % American Mathematical Society packages for advanced math
\usepackage{graphicx}                   % For including images
\usepackage{fancyhdr}                   % For creating custom headers and footers
\usepackage[colorlinks=true, urlcolor=blue, linkcolor=blue]{hyperref} % For clickable links
\usepackage{cancel}
\usepackage{array}
\usepackage{amsfonts}
\usepackage{amsxtra}
\usepackage{epsfig}
\usepackage{wasysym}
\usepackage{relsize}
\usepackage{tikz}
\tikzset{every picture/.style={scale=1.2}}
\renewcommand{\normalsize}{\fontsize{12}{20}\selectfont}

% custom commands
\newcommand{\myauthor}{Miguel Gomez}
\newcommand{\canceling}[2]{\textcolor{red}{\cancelto{\textcolor{black}{#1}}{\textcolor{black}{#2}}}}
\newcommand{\todo}[1]{\textcolor{blue}{TODO:#1}}
\newcommand{\DNEpaths}[1]{\boxed{$\therefore$ The limit DNE #1 because different paths give different results.}}

% --- DOCUMENT & AUTHOR INFORMATION ---
\title{Homework \# 3: }
\author{
	MATH 3160 -- Complex Variables\\
	\myauthor
}
\date{Completed: \today}

% --- HEADER & FOOTER CONFIGURATION ---
% This section sets up the header that will appear on each page.
\pagestyle{fancy}
\fancyhf{} % Clears the default header and footer
\lhead{Math 3160 -- HW \# 3} % Left side of header
\rhead{\myauthor} % Puts the author's name on the right side
\rfoot{Page \thepage} % Puts the page number on the bottom right

\begin{document}

\maketitle % This command generates the title based on the information above.

% ====================================================================
% --- START OF PROBLEMS ---
% ====================================================================

\section*{Problem 1: }
\begin{enumerate}
	\item  [(a)]


	      Write the function
	      \[
		      f(z)=z+\frac{1}{z}\qquad (z\neq 0)
	      \]
	      in the form $f(z)=u(r,\theta)+iv(r,\theta)$.

	\item [(b)] Show that the image of the points in the upper half plane ($y>0$) that are exterior to the circle $|z|=1$ are mapped under $f$ to the entire upper half plane $v>0$.
\end{enumerate}

\subsection*{(a)}
\begin{align*}
	f(z) & =z+\frac{1}{z} =  (x+iy)+\frac{1}{(x+iy)}  = \frac{(x+iy)(x^2+y^2)}{(x^2+y^2)} + \frac{x-iy}{(x^2+y^2)}        \\
	     & = \frac{1}{x^2+y^2}((x+iy)(x^2+y^2) + x-iy) = \frac{1}{x^2+y^2}(x(x^2+y^2) + x + i(y(x^2+y^2) - y))            \\
	     & \therefore u(x,y) = \frac{1}{x^2+y^2}(x(x^2+y^2) + x)\quad \& \quad v(x,y) = \frac{1}{x^2+y^2}(y(x^2+y^2) - y) \\
	     & r^2 = x^2 + y^2                                                                                                \\
	     & x = r\cos{(\theta)}                                                                                            \\
	     & y = r\sin{(\theta)}
\end{align*}
\begin{align*}
	\therefore u(r,\theta) = \frac{1}{r^2}(r^3\cos{(\theta)} + r\cos{(\theta)}) & = r\cos{(\theta)} + \frac{1}{r}\cos{(\theta)}        \\
	                                                                            & = \boxed{\left(r + \frac{1}{r}\right)\cos{(\theta)}} \\
	\quad \ v(r,\theta) = \frac{1}{r^2}(r^3\sin{(\theta)} - r\sin{(\theta)})    & = r\sin{(\theta)} - \frac{1}{r}\sin{(\theta)}        \\
	                                                                            & = \boxed{\left(r - \frac{1}{r}\right)\sin{(\theta)}}
\end{align*}
\subsection*{(b)}
Since we have $r > 1$ for any point exterior to $|z| = 1$, then the condition for the $y$ value:
\begin{align*}
	v(r,\theta) & = \left(r - \frac{1}{r}\right)\sin{(\theta)}                                    \\
	            & = \left(r - \frac{1}{r}\right) > 0 \quad \forall\ r > 1                         \\
	            & = 0 \leq \sin{(\theta)} \leq 1 \quad \forall\ \theta\ |\ 0 \leq \theta \leq \pi
\end{align*}
Meaning all positive values for $y=r\sin{(\theta)}$ map to positive values for $v(r,\theta)$. Since a point inside the circle $|z|= 1$ would have $r \leq 1$, that would put the value of the factor on $\sin{(\theta)}$ less than $0$. Taking the simplest case of $r = 1+\epsilon$ where $\epsilon$ is a small increase, and the angle $\theta = 0$, we get a point $v$ that is equal to $0$. As we sweep the angle to $\frac{\pi}{2}$, we get a factor of $1$ multiplied by something larger than 0.
\begin{align*}
	v(r,\theta) & = \left(r - \frac{1}{r}\right)\sin{(\theta)} = \left((r+\epsilon) - \frac{1}{(r+\epsilon)}\right)\sin{(\theta)}                                                                                       \\
	            & =\left(\frac{(r+\epsilon)^2}{(r+\epsilon)} - \frac{1}{r+\epsilon}\right)\sin{(\theta)}=\left(\frac{(r+\epsilon)^2-1}{r+\epsilon}\right)\sin{(\theta)}                                                 \\
	            & =\left(\frac{((r+\epsilon)^2-1)(r-\epsilon)}{(r+\epsilon)(r-\epsilon)}\right)\sin{(\theta)}=\left(\frac{(r^2+2r\epsilon + \epsilon^2 -1)(r-\epsilon)}{r^2-\epsilon^2}\right)\sin{(\theta)}            \\
	            & =\left(\frac{(r^2+2r\epsilon + \canceling{0}{\epsilon^2} -1)(r-\epsilon)}{r^2-\canceling{0}{\epsilon^2}}\right)\sin{(\theta)}=\left(\frac{(r^2+2r\epsilon  -1)(r-\epsilon)}{r^2}\right)\sin{(\theta)} \\
	            & =\left(\frac{(r^2+2r\epsilon  -1)(r-\epsilon)}{r^2}\right)\sin{(\theta)} =\left(\frac{(r^3+2r^2\epsilon -r -r^2\epsilon-2r\epsilon^2+\epsilon)}{r^2}\right)\sin{(\theta)}                             \\
	            & =\left(\frac{(r^3+r^2\epsilon -r -\canceling{0}{2r\epsilon^2}+\epsilon)}{r^2}\right)\sin{(\theta)}
	=\left(\frac{(r^3+r^2\epsilon -r +\epsilon)}{r^2}\right)\sin{(\theta)}                                                                                                                                              \\
	            & =\left(r + \epsilon - \frac{1}{r} + \frac{\epsilon}{r^2}\right)\sin{(\theta)} = \left(\left(r - \frac{1}{r}\right)+ \left(\epsilon  + \frac{\epsilon}{r^2}\right)\right)\sin{(\theta)}
\end{align*}
With $r = 1$ with the very small increase, we can plug $1$ in for $r$ and we see that the overall value is still positive.
\begin{align*}
	 & \left(\left(r - \frac{1}{r}\right)+ \left(\epsilon  + \frac{\epsilon}{r^2}\right)\right)\sin{(\theta)} = \left(\left(1 - \frac{1}{1}\right)+ \left(\epsilon  + \frac{\epsilon}{1}\right)\right)\sin{(\theta)} \\
	 & \left((1-1)+ (2\epsilon)\right)\sin{(\theta)} = 2\epsilon\sin{(\theta)}
\end{align*}
\boxed{$\therefore$ all values in the upper half plane outside the circle $|z|= 1$ will map to the values such that $v > 0$.}
\todo{Showing the condition maps to the entire half-plane is not yet shown, but I think this is a decent argument so far. Come up with a method for showing this condition satisfies and covers the entire half-plane.}
%   \begin{center}
% \begin{minipage}{0.45\textwidth}
%     \centering
%     \begin{tikzpicture}
%         \draw[->] (-2,0) -- (3,0) node[right] {$\mathbb{R}$};
%         \draw[->] (0,-2) -- (0,3) node[above] {$\mathbb{I}$};
%         % Add your content for left graph here
%     \end{tikzpicture}

%     \text{$z$-plane}
% \end{minipage}
% \hfill
% \begin{minipage}{0.45\textwidth}
%     \centering
%     \begin{tikzpicture}
%         \draw[->] (-2,0) -- (3,0) node[right] {$\mathbb{R}$};
%         \draw[->] (0,-2) -- (0,3) node[above] {$\mathbb{I}$};
%         % Add your content for right graph here
%     \end{tikzpicture}

%     \text{$f(z)$-plane}
% \end{minipage}
% \end{center}

\section*{Problem 2: }
Use the rectangular forms or exponential forms for the following functions to prove that
\begin{enumerate}
\item[(a)]  $\lim\limits_{z\to z_0} Re(z) = Re(z_0)$
  \begin{align*}
    \lim\limits_{z\to z_0} Re(z) &= \lim\limits_{x\to x_0} x = x_0
  \end{align*}
  We can stop here since this is sufficient.
  
\item[(b)] $\lim\limits_{z\to z_0} \bar{z} = \bar{z_0} $
  \begin{align*}
    \lim\limits_{z\to z_0} \bar{z} = \lim\limits_{(x,y)\to (x_0,y_0)} \overline{(x+iy)} &= \lim\limits_{(x,y)\to (x_0,y_0)}(x-iy) \\
    &=\lim\limits_{(x,y)\to (x_0,y_0)} = (x_0-iy_0)
  \end{align*}
  We can stop here since this is sufficient.
\item[(c)] $\lim\limits_{z\to 0} \frac{\bar{z}^2}{z} = 0$
  \begin{align*}
    \lim\limits_{z\to 0} \frac{\bar{z}^2}{z} &= \lim\limits_{r\to 0} \frac{\overline{re^{i\theta}}^2}{re^{i\theta}}\\
                                             &= \lim\limits_{z\to 0} \frac{(re^{-i\theta})^2}{re^{i\theta}} = \lim\limits_{r\to 0}\frac{r^2e^{-i2\theta}}{re^{i\theta}}\\
                                             &= \lim\limits_{r\to 0}r\frac{e^{-i2\theta}}{e^{i\theta}} = \lim\limits_{r\to 0}re^{-i2\theta}e^{-i\theta} \\
    &= \lim\limits_{r\to 0}re^{-i3\theta}
  \end{align*}
  For this, no matter the angle used, we still have a dependence on $r$ in the expression, so from any path, we will approach $0$.
\end{enumerate}

% For graphs/diagrams, you can use TikZ:
% \begin{center}
% 	\begin{tikzpicture}
% 		% Your TikZ code here
% 		% Example: Draw axes
% 		\draw[->] (-2,0) -- (3,0) node[right] {$\mathbb{R}$};
% 		\draw[->] (0,-2) -- (0,3) node[above] {$\mathbb{I}$};
% 	\end{tikzpicture}
% \end{center}

\vspace{.5cm} % Space between problems

\hrule

\newpage
\section*{Problem 3: }
Show that the limit of the function
\[ f(z) =\Big( \frac{z}{\bar{z}} \Big)^3\]
as $z$ tends to zero does not exist. Do so by examining several test paths going to zero.

\begin{align*}
  z &= re^{i\theta}\\
  \bar z &= re^{-i\theta} \\
  \frac{z}{\bar z} = \frac{re^{i\theta}}{re^{-i\theta}} &= re^{i\theta}r^{-1}e^{i\theta} = e^{i2\theta} \\
  \left(\frac{z}{\bar z}\right)^3 = (e^{i2\theta})^3 &= e^{i6\theta}
\end{align*}
We can see that the dependence on $r$ is now gone as the two canceled out. Evaluating two paths with differing value of $\theta$ will give two different magnitudes for the result. First with $\theta = 0$
\begin{align*}
  \lim\limits_{r \to 0}e^{i6\theta} = \lim\limits_{r \to 0}e^{0} = \lim\limits_{r \to 0} = 1
\end{align*}
as in it becomes $1$ from the positive side. Now with $\theta= \frac{\pi}{2}$: 
\begin{align*}
  \lim\limits_{r \to 0}e^{i6\theta} = \lim\limits_{r \to 0}e^{i6\frac{\pi}{2}} = \lim\limits_{r \to 0}e^{i3\pi} = -1
\end{align*}
Here it becomes $-1$ by approaching from the negative side.

\DNEpaths

\vspace{.5cm} % Space for work

\hrule

% Add more problems as needed...
\newpage
\section*{Problem 4: }
Does $f(x+iy)=\displaystyle{\frac{x+iy}{x+2iy}}$ have a limit as $x+iy \to 0$\,?

No the function does not have a limit as it tends to $0$ because the expression $x+iy$ can only become $0$ if both $x$ and $y$ tend to $0$:
\begin{align*}
  \frac{x+iy}{x+2iy} &= \frac{x+iy}{x+iy + iy} = \frac{x+iy}{(x+iy) + iy} \neq \frac{0}{iy}
\end{align*}
This cannot equal $0$ because if $x+iy$ implies that both $x$ and $y$ tend to $0$, then that implies that the denominator is also going to become $0$ which we know is an indeterminate form. Evaluating from two paths again as we have done previously. We can start with $y=0$ and evaluate the limit from the real axis.
\begin{align*}
  \frac{x+iy}{x+2iy} &= \frac{x+i(0)}{x+2i(0)} = \frac{x}{x} = 1
\end{align*}
This loses all dependence on the variables and is $1$ for any values approaching $0$. Approaching now from the imaginary axis.
\begin{align*}
  \frac{x+iy}{x+2iy} &= \frac{(0)+iy}{(0)+2iy} =\frac{iy}{2iy} =\frac{1}{2} 
\end{align*}
This loses all dependence on the variables and is $\frac{1}{2}$ for any values approaching $0$.

\DNEpaths

\vspace{.5cm} % Space for work

\hrule
\end{document}

%%% Local Variables:
%%% mode: latex
%%% TeX-master: t
%%% End:
