% --------------------------------------------------------------------
% LaTeX Template for Math Homework
% --------------------------------------------------------------------

\documentclass{article}

% --- PACKAGE IMPORTS ---
% These packages add functionality for math symbols, formatting, etc.
\usepackage[margin=.7in]{geometry}       % For setting page margins
\usepackage{amsmath, amssymb, amsthm}   % American Mathematical Society packages for advanced math
\usepackage{graphicx}                   % For including images
\usepackage{fancyhdr}                   % For creating custom headers and footers
\usepackage[colorlinks=true, urlcolor=blue, linkcolor=blue]{hyperref} % For clickable links
\usepackage{cancel}
\usepackage{array}
\usepackage{amsfonts}
\usepackage{amsxtra}
\usepackage{epsfig}
\usepackage{wasysym}
\usepackage{relsize}
\usepackage{tikz}
\tikzset{every picture/.style={scale=1.2}}
\renewcommand{\normalsize}{\fontsize{12}{20}\selectfont}

% custom commands
\newcommand{\myauthor}{Miguel Gomez}
\newcommand{\canceling}[2]{\textcolor{red}{\cancelto{\textcolor{black}{#1}}{\textcolor{black}{#2}}}}
\newcommand{\todo}[1]{\textcolor{blue}{TODO:#1}}

% --- DOCUMENT & AUTHOR INFORMATION ---
\title{Homework \# 3: }
\author{
	MATH 3160 -- Complex Variables\\
	\myauthor
}
\date{Completed: \today}

% --- HEADER & FOOTER CONFIGURATION ---
% This section sets up the header that will appear on each page.
\pagestyle{fancy}
\fancyhf{} % Clears the default header and footer
\lhead{Math 3160 -- HW \# 3} % Left side of header
\rhead{\myauthor} % Puts the author's name on the right side
\rfoot{Page \thepage} % Puts the page number on the bottom right

\begin{document}

\maketitle % This command generates the title based on the information above.

% ====================================================================
% --- START OF PROBLEMS ---
% ====================================================================

\section*{Problem 1: }
\begin{enumerate}
	\item  [(a)]


	      Write the function
	      \[
		      f(z)=z+\frac{1}{z}\qquad (z\neq 0)
	      \]
	      in the form $f(z)=u(r,\theta)+iv(r,\theta)$.

	\item [(b)] Show that the image of the points in the upper half plane ($y>0$) that are exterior to the circle $|z|=1$ are mapped under $f$ to the entire upper half plane $v>0$.
\end{enumerate}

\subsection*{(a)}
\begin{align*}
  f(z)&=z+\frac{1}{z} =  (x+iy)+\frac{1}{(x+iy)}  = \frac{(x+iy)(x^2+y^2)}{(x^2+y^2)} + \frac{x-iy}{(x^2+y^2)}\\
      &= \frac{1}{x^2+y^2}((x+iy)(x^2+y^2) + x-iy) = \frac{1}{x^2+y^2}(x(x^2+y^2) + x + i(y(x^2+y^2) - y))\\
      & \therefore u(x,y) = \frac{1}{x^2+y^2}(x(x^2+y^2) + x)\quad \& \quad v(x,y) = \frac{1}{x^2+y^2}(y(x^2+y^2) - y)\\
      & r^2 = x^2 + y^2\\
      & x = r\cos{(\theta)}\\
      & y = r\sin{(\theta)}
\end{align*}
\begin{align*}
  \therefore u(r,\theta) = \frac{1}{r^2}(r^3\cos{(\theta)} + r\cos{(\theta)}) &= r\cos{(\theta)} + \frac{1}{r}\cos{(\theta)}\\
                                                                              &= \boxed{\left(r + \frac{1}{r}\right)\cos{(\theta)}}\\
  \quad \ v(r,\theta) = \frac{1}{r^2}(r^3\sin{(\theta)} - r\sin{(\theta)}) &= r\sin{(\theta)} - \frac{1}{r}\sin{(\theta)}\\
                                                                              &= \boxed{\left(r - \frac{1}{r}\right)\sin{(\theta)}}
\end{align*}
\subsection*{(b)}
  \begin{center}
\begin{minipage}{0.45\textwidth}
    \centering
    \begin{tikzpicture}
        \draw[->] (-2,0) -- (3,0) node[right] {$\mathbb{R}$};
        \draw[->] (0,-2) -- (0,3) node[above] {$\mathbb{I}$};
        % Add your content for left graph here
    \end{tikzpicture}
    
    \text{$z$-plane}
\end{minipage}
\hfill
\begin{minipage}{0.45\textwidth}
    \centering
    \begin{tikzpicture}
        \draw[->] (-2,0) -- (3,0) node[right] {$\mathbb{R}$};
        \draw[->] (0,-2) -- (0,3) node[above] {$\mathbb{I}$};
        % Add your content for right graph here
    \end{tikzpicture}
    
    \text{$f(z)$-plane}
\end{minipage}
\end{center}


\hrule % Adds a horizontal line to separate problems.

\newpage
\section*{Problem 2: }
Use the rectangular forms or exponential forms for the following functions to prove that
		\begin{enumerate}
			\item[(a)]  $\lim\limits_{z\to z_0} Re(z) = Re(z_0)$
			\item[(b)] $\lim\limits_{z\to z_0} \bar{z} = \bar{z_0} $
			\item[(c)] $\lim\limits_{z\to 0} \frac{\bar{z}^2}{z} = 0$
		\end{enumerate}

% For graphs/diagrams, you can use TikZ:
% \begin{center}
% 	\begin{tikzpicture}
% 		% Your TikZ code here
% 		% Example: Draw axes
% 		\draw[->] (-2,0) -- (3,0) node[right] {$\mathbb{R}$};
% 		\draw[->] (0,-2) -- (0,3) node[above] {$\mathbb{I}$};
% 	\end{tikzpicture}
% \end{center}

\vspace{.5cm} % Space between problems

\hrule

\newpage
\section*{Problem 3: }
Show that the limit of the function 
		\[ f(z) =\Big( \frac{z}{\bar{z}} \Big)^3\]
		as $z$ tends to zero does not exist. Do so by examining several test paths going to zero.
		
\vspace{.5cm} % Space for work

\hrule

% Add more problems as needed...
\newpage
\section*{Problem 4: }
Does $f(x+iy)=\displaystyle{\frac{x+iy}{x+2iy}}$ have a limit as $x+iy \to 0$\,?

\vspace{.5cm} % Space for work

\hrule
\end{document}

%%% Local Variables:
%%% mode: latex
%%% TeX-master: t
%%% End:
