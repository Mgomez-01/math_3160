% --------------------------------------------------------------------
% LaTeX Template for Math Homework
% --------------------------------------------------------------------

\documentclass{article}

% --- PACKAGE IMPORTS ---
% These packages add functionality for math symbols, formatting, etc.
\usepackage[margin=.7in]{geometry}       % For setting page margins
\usepackage{amsmath, amssymb, amsthm}   % American Mathematical Society packages for advanced math
\usepackage{graphicx}                   % For including images
\usepackage{fancyhdr}                   % For creating custom headers and footers
\usepackage[colorlinks=true, urlcolor=blue, linkcolor=blue]{hyperref} % For clickable links
\usepackage{cancel}
\usepackage{array}
\usepackage{amsfonts}
\usepackage{amsxtra}
\usepackage{epsfig}
\usepackage{wasysym}
\usepackage{relsize}
\usepackage{tikz}
\tikzset{every picture/.style={scale=1.2}}
\renewcommand{\normalsize}{\fontsize{12}{20}\selectfont}

% custom commands
\newcommand{\myauthor}{Your Name Here}
\newcommand{\canceling}[2]{\textcolor{red}{\cancelto{\textcolor{black}{#1}}{\textcolor{black}{#2}}}}
\newcommand{\todo}[1]{\textcolor{blue}{TODO:#1}}

% --- DOCUMENT & AUTHOR INFORMATION ---
\title{Homework \#2}
\author{
  MATH 3160 -- complex variables\\
  \myauthor
}
\date{Completed: \today}

% --- HEADER & FOOTER CONFIGURATION ---
% This section sets up the header that will appear on each page.
\pagestyle{fancy}
\fancyhf{} % Clears the default header and footer
\lhead{Math 3160 -- HW \#2} % Left side of header
\rhead{\myauthor} % Puts the author's name on the right side
\rfoot{Page \thepage} % Puts the page number on the bottom right

\begin{document}

\maketitle % This command generates the title based on the information above.

% ====================================================================
% --- START OF PROBLEMS ---
% ====================================================================

\section*{Problem 1}
By writing the individual factors on the left in exponential form, performing the needed
operations, and finally changing back to rectangular coordinates, show that

\begin{enumerate}
\item[(a)] $i(1-\sqrt{3}i)(\sqrt{3} + i) = 2(1 + \sqrt{3}i)$
  \begin{align*}
    % Your work here
  \end{align*}
\hrule
\item[(b)] $\frac{5i}{2+i} = 1+2i$
  \begin{align*}
    % Your work here
  \end{align*}
\hrule
\item[(c)] $(-1 + i)^7 = -8(1+i)$
  \begin{align*}
    % Your work here
  \end{align*}
\hrule
\item[(d)] $(1 + \sqrt{3}i)^{-10}= 2^{-11}(-1 + \sqrt{3}i)$
  \begin{align*}
    % Your work here
  \end{align*}
\hrule
\end{enumerate}

\vspace{.5cm} % Space between problems


\newpage
\section*{Problem 2}
Find the square roots of (a) $2i$ and (b) $(1-\sqrt{3}i)$ express them in rectangular coordinates
\newpage
\section*{Problem 3}
Find all roots and indicate in rectangular coordinates
\begin{enumerate}
\item[(a)] $(-16)^{\frac{1}{4}}$
  \begin{align*}
    % Your work here
  \end{align*}
\hrule
\item[(b)] $(-8-8\sqrt{3}i)^{\frac{1}{4}}$
  \begin{align*}
    % Your work here
  \end{align*}
\hrule
\end{enumerate}


\newpage

\section*{Problem 4}
 Find the four zeros of $z^4+4$
\begin{align*}
  
\end{align*}
 \vspace{1cm}
 \hrule
\newpage

\section*{Problem 5}
Show that if $c$ is an $n^{\text{th}}$ root of 1 other than 1 itself, then:
\begin{align*}
  1 + c + c^2 + &... + c^{n-1} = 0
\end{align*}
Hint: multiply above by $(c-1)$

multiplying the above by $(c-1)$ gives the following
\begin{align*}
  (c-1)\cdot(1 + c + c^2 + &... + c^{n-1}) = (c-1)\cdot 0\\
  c + c^2 + c^3 + &... + c^{n-1} + c^n +\ \ \ \text{ expanding }c \\
  -1 -c -c^2 -&... -c^{n-1} \ \ \ \text{expanding }-1\\
  -1 + (c-c) + (c^2-c^2) + (c^3-c^3) +&... + (c^{n-1}-c^{n-1}) + c^n = 0\\
  -1 + \canceling{0}{(c-c)} + \canceling{0}{(c^2-c^2)} + \canceling{0}{(c^3-c^3)} +&... + \canceling{0}{(c^{n-1}-c^{n-1})} + c^n = 0\\
  -1 + c^n &= 0\\
  c^n &= 1\\
  \sqrt[n]{c^n} &= \sqrt[n]{1}\\
  c &= 1\\
\end{align*}
However, if this is the case and $c = 1$, then the lhs should equal 0 when we plug in 1 for $c$:
\begin{align*}
  1 + (1) + (1)^2 + &... + (1)^{n-1} = 0\\
  1 + 1 + 1 + &... + 1 = n \neq 0\\
\end{align*}
This appears to be a contradiction. Now using something other than 1, i.e. $c\neq 1$, then the sum cannot be 0.
\begin{align*}
  1 + c + c^2 + &... + c^{n-1} = S\\
  \text{same steps as before}&\\
  -1 + c^n &= S(c-1) = Sc - S\\
  c^n - Sc + S - 1 &= 0 \\
\end{align*}

 \vspace{1cm}
 \hrule
\newpage

\section*{Problem 6}
For each of the below, indicate the domain of definition.
\begin{enumerate}
  \item[(a)] $f(z) = \frac{1}{z^2+1}$
\begin{align*}
  
\end{align*}
 \vspace{1cm}
 \hrule
  \item[(b)] $f(z) = \text{Arg}\left(\frac{1}{z}\right)$
\begin{align*}
  
\end{align*}
 \vspace{1cm}
 \hrule
  \item[(c)] $f(z) = \frac{z}{z + \bar{z}}$
\begin{align*}
  
\end{align*}
 \vspace{1cm}
 \hrule
  \item[(d)] $f(z) = \frac{1}{(1-|z|^2)}$
\end{enumerate}
\newpage
\section{Problem 7}
 Sketch the region onto which the sector $r \le 1;\ 0 \le \theta \le \frac{\pi}{4}$ in the $z$-plane is mapped to
 the $w = f(z)$-plane by the transformations
\begin{enumerate}
  \item[(a)] $w = z^2$
% Method 2: Two separate TikZ environments using minipages

\begin{center}
\begin{minipage}{0.45\textwidth}
    \centering
    \begin{tikzpicture}
        \draw[->] (-2,0) -- (3,0) node[right] {$\mathbb{R}$};
        \draw[->] (0,-2) -- (0,3) node[above] {$\mathbb{I}$};
        % Add your content for left graph here
    \end{tikzpicture}
    
    \text{$z$-plane}
\end{minipage}
\hfill
\begin{minipage}{0.45\textwidth}
    \centering
    \begin{tikzpicture}
        \draw[->] (-2,0) -- (3,0) node[right] {$\mathbb{R}$};
        \draw[->] (0,-2) -- (0,3) node[above] {$\mathbb{I}$};
        % Add your content for right graph here
    \end{tikzpicture}
    
    \text{$w$-plane}
\end{minipage}
\end{center}
\newpage
\item[(b)] $w = z^3$
  \begin{center}
\begin{minipage}{0.45\textwidth}
    \centering
    \begin{tikzpicture}
        \draw[->] (-2,0) -- (3,0) node[right] {$\mathbb{R}$};
        \draw[->] (0,-2) -- (0,3) node[above] {$\mathbb{I}$};
        % Add your content for left graph here
    \end{tikzpicture}
    
    \text{$z$-plane}
\end{minipage}
\hfill
\begin{minipage}{0.45\textwidth}
    \centering
    \begin{tikzpicture}
        \draw[->] (-2,0) -- (3,0) node[right] {$\mathbb{R}$};
        \draw[->] (0,-2) -- (0,3) node[above] {$\mathbb{I}$};
        % Add your content for right graph here
    \end{tikzpicture}
    
    \text{$w$-plane}
\end{minipage}
\end{center}
 \vspace{1cm}
 \newpage
\item[(c)] $w = z^4$
  \begin{center}
\begin{minipage}{0.45\textwidth}
    \centering
    \begin{tikzpicture}
        \draw[->] (-2,0) -- (3,0) node[right] {$\mathbb{R}$};
        \draw[->] (0,-2) -- (0,3) node[above] {$\mathbb{I}$};
        % Add your content for left graph here
    \end{tikzpicture}
    
    \text{$z$-plane}
\end{minipage}
\hfill
\begin{minipage}{0.45\textwidth}
    \centering
    \begin{tikzpicture}
        \draw[->] (-2,0) -- (3,0) node[right] {$\mathbb{R}$};
        \draw[->] (0,-2) -- (0,3) node[above] {$\mathbb{I}$};
        % Add your content for right graph here
    \end{tikzpicture}
    
    \text{$w$-plane}
\end{minipage}
\end{center}

\end{enumerate}
% Add more problems as needed...
\newpage
% For graphs/diagrams, you can use TikZ:
\begin{center}
\begin{tikzpicture}
    % Your TikZ code here
    % Example: Draw axes
    \draw[->] (-2,0) -- (3,0) node[right] {$\mathbb{R}$};
    \draw[->] (0,-2) -- (0,3) node[above] {$\mathbb{I}$};
\end{tikzpicture}
\end{center}

%example two planes side by side with minipage
  \begin{center}
\begin{minipage}{0.45\textwidth}
    \centering
    \begin{tikzpicture}
        \draw[->] (-2,0) -- (3,0) node[right] {$\mathbb{R}$};
        \draw[->] (0,-2) -- (0,3) node[above] {$\mathbb{I}$};
        % Add your content for left graph here
    \end{tikzpicture}
    
    \text{$z$-plane}
\end{minipage}
\hfill
\begin{minipage}{0.45\textwidth}
    \centering
    \begin{tikzpicture}
        \draw[->] (-2,0) -- (3,0) node[right] {$\mathbb{R}$};
        \draw[->] (0,-2) -- (0,3) node[above] {$\mathbb{I}$};
        % Add your content for right graph here
    \end{tikzpicture}
    
    \text{$w$-plane}
\end{minipage}
\end{center}


\end{document}

%%% Local Variables:
%%% mode: latex
%%% TeX-master: t
%%% End:
