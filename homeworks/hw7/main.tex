% --------------------------------------------------------------------
% LaTeX Template for Math Homework
% --------------------------------------------------------------------

\documentclass{article}

% --- PACKAGE IMPORTS ---
% These packages add functionality for math symbols, formatting, etc.
\usepackage[margin=.7in]{geometry}       % For setting page margins
\usepackage{amsmath, amssymb, amsthm}   % American Mathematical Society packages for advanced math
\usepackage{graphicx}                   % For including images
\usepackage{fancyhdr}                   % For creating custom headers and footers
\usepackage[colorlinks=true, urlcolor=blue, linkcolor=blue]{hyperref} % For clickable links
\usepackage{cancel}
\usepackage{array}
\usepackage{amsfonts}
\usepackage{amsxtra}
\usepackage{epsfig}
\usepackage{wasysym}
\usepackage{relsize}
\usepackage{tikz}
\tikzset{every picture/.style={scale=1.2}}
\renewcommand{\normalsize}{\fontsize{12}{20}\selectfont}

% custom commands
\newcommand{\myauthor}{Miguel Gomez}
\newcommand{\canceling}[2]{\textcolor{red}{\cancelto{\textcolor{black}{#1}}{\textcolor{black}{#2}}}}
\newcommand{\todo}[1]{\textcolor{blue}{TODO:#1}}
% Save the original commands
\let\oldcos\cos
\let\oldsin\sin
\let\oldcosh\cosh
\let\oldsinh\sinh

% Redefine with automatic parentheses
\renewcommand{\cos}[1]{\oldcos\left(#1\right)}
\renewcommand{\sin}[1]{\oldsin\left(#1\right)}
\renewcommand{\cosh}[1]{\oldcosh\left(#1\right)}
\renewcommand{\sinh}[1]{\oldsinh\left(#1\right)}

\newcommand{\der}[2]{\frac{d#1}{d#2}}
\newcommand{\secder}[2]{\frac{d^2#1}{d#2^2}}
\newcommand{\parder}[2]{\frac{\partial#1}{\partial#2}}
\newcommand{\secparder}[2]{\frac{\partial^2#1}{\partial#2^2}}

% --- DOCUMENT & AUTHOR INFORMATION ---
\title{Homework \# 7}
\author{
  MATH 3160 -- Complex Variables\\
  \myauthor
}
\date{Completed: \today}

% --- HEADER & FOOTER CONFIGURATION ---
% This section sets up the header that will appear on each page.
\pagestyle{fancy}
\fancyhf{} % Clears the default header and footer
\lhead{Math 3160 -- HW \# 7} % Left side of header
\rhead{\myauthor} % Puts the author's name on the right side
\rfoot{Page \thepage} % Puts the page number on the bottom right

\begin{document}

\maketitle % This command generates the title based on the information above.

% ====================================================================
% --- START OF PROBLEMS ---
% ====================================================================

\section*{Problem 1}
Compute the integral \[ \int_{C_i} z^{1/2} dz\] where $ z^{1/2} $ is taken on the branch $0<arg(z)<2\pi$ and along the contours $C_1 , C_2$,  where $C_1$ is any contour from $-3$ to $3$ lying in the domain except the point $3$ that approaches $3$ from above the real axis, while $C_2$ approaches $3$ from below the real axis.
               
\vspace{.5cm} % Space between problems

\hrule % Adds a horizontal line to separate problems.

\newpage
\section*{Problem 2}
Apply the Cauchy-Goursat theorem to show that \[  \int_C f(z)dz = 0\] for the contour $C$ being the unit circle about the origin, or determine that Cauchy-Goursat theorem does not apply.

\begin{enumerate}
\item[(a)] $f(z) = \frac{z^2}{z-3}$
\item[(b)] $f(z) = ze^{-z}$
\item[(c)] $f(z) = \frac{z}{2z-i}$ 
\item[(d)] $f(z) = \tan(z)$
\item[(e)] $f(z) = Log(z+2)$ 
\item[(f)] $f(z) = \frac{1}{z^2+3z+2}$
\item[(g)] $f(z) = log(z)$, any branch.
\end{enumerate}
		
\vspace{.5cm} % Space between problems

\hrule

\newpage
\section*{Problem 3}
 Let C denote the positively oriented boundary of the square whose sides lie along $x\pm2$ and $y\pm2$. Evaluate the following integrals.
 
 \begin{enumerate}
 \item[(a)] $\int_C \frac{e^{-z}}{z-\frac{\pi i}{2}}dz$
 \item[(b)] $\int_C \frac{z}{2z+1}dz$
 \item[(c)] $\int_C \frac{\cosh{z}}{z}$
 \end{enumerate}
 
\vspace{.5cm} % Space for work

\hrule
\newpage
\section*{Problem 4}
Use the Cauchy integral formula to integrate $\int_C f(z)dz$ for $f(z) = \frac{1}{(z-i)(z-1)}$ over a contour $C$ being a positive-oriented circle at the origin with radius 2. 
	
\vspace{.5cm} % Space for work

\hrule
\newpage
\section*{Problem 5}
Using the extended Cauchy integral formula, compute \[ \int_C \frac{e^{2i z}}{(z-3i)^4} dz \] on the curve $C$ being a positive-oriented circle with radius 4 centered at the origin.
	

\vspace{.5cm} % Space for work

\hrule
\newpage
\section*{Problem 6}
Using the extended Cauchy integral formula, compute \[ \int_C \frac{\cosh{z}}{(z-i)^2} dz \] on the contour $C$ being the positive-oriented circle with radius 2 centered at the origin. Recall that $\cosh{z} := \frac{e^z + e^{-z}}{2}.$
	

\end{document}

%%% Local Variables:
%%% mode: latex
%%% TeX-master: t
%%% End:
