% --------------------------------------------------------------------
% LaTeX Template for Math Homework
% --------------------------------------------------------------------

\documentclass{article}

% --- PACKAGE IMPORTS ---
% These packages add functionality for math symbols, formatting, etc.
\usepackage[margin=.7in]{geometry}       % For setting page margins
\usepackage{amsmath, amssymb, amsthm}   % American Mathematical Society packages for advanced math
\usepackage{graphicx}                   % For including images
\usepackage{fancyhdr}                   % For creating custom headers and footers
\usepackage[colorlinks=true, urlcolor=blue, linkcolor=blue]{hyperref} % For clickable links
\usepackage{cancel}
\usepackage{array}
\usepackage{amsfonts}
\usepackage{amsxtra}
\usepackage{epsfig}
\usepackage{wasysym}
\usepackage{relsize}
\usepackage{tikz}
\tikzset{every picture/.style={scale=1.2}}
\renewcommand{\normalsize}{\fontsize{12}{20}\selectfont}

% custom commands
\newcommand{\myauthor}{Your Name Here}
\newcommand{\canceling}[2]{\textcolor{red}{\cancelto{\textcolor{black}{#1}}{\textcolor{black}{#2}}}}
\newcommand{\todo}[1]{\textcolor{blue}{TODO:#1}}

% --- DOCUMENT & AUTHOR INFORMATION ---
\title{Homework \# 4}
\author{
	MATH 3160 -- Complex Variables\\
	\myauthor
}
\date{Completed: \today}

% --- HEADER & FOOTER CONFIGURATION ---
% This section sets up the header that will appear on each page.
\pagestyle{fancy}
\fancyhf{} % Clears the default header and footer
\lhead{Math 3160 -- HW \# 4} % Left side of header
\rhead{\myauthor} % Puts the author's name on the right side
\rfoot{Page \thepage} % Puts the page number on the bottom right

\begin{document}

\maketitle % This command generates the title based on the information above.

% ====================================================================
% --- START OF PROBLEMS ---
% ====================================================================

\section*{Problem 1}
Find $f'(z)$ using differentiation rules.
\begin{enumerate}
	\item[(a)]  $f(z) = 3z^2-2z+4$
	\item[(b)] $f(z) = (1-4z)^3$
	\item[(c)] $f(z) = \frac{z-1}{2z+1}$, assume $z\neq -1/2$
	\item[(d)] $f(z) = \frac{(z^2+1)^4}{z^2}$, assume $z\neq 0$
	\item[(e)] $f(z) = z \, e^{z^2 + 3}$.
\end{enumerate}


\vspace{.5cm} % Space between problems

\hrule % Adds a horizontal line to separate problems.

\newpage
\section*{Problem 2}
Show that $f'(z_0)$ does not exist at any point $z_0$ in two ways: using the limit definition and using the Cauchy-Riemann equations. Here, $z = x + iy$ and $x,y \in \mathbb{R}$.
		\begin{enumerate}
			\item[(a)] $f(z) = 2x+ixy^2$
			\item[(b)] $f(z) = e^{x}e^{-iy}$
		\end{enumerate}
	

\vspace{.5cm} % Space between problems

\hrule

\newpage
\section*{Problem 3}
Using the exponential function $e^{z}$, we can now define the complex cosine and sine function for any $z \in \mathbb{C}$ as follows:
		\[  \cos(z) = \frac{e^{iz} + e^{-iz}}{2} \]
		
		and 
		\[  \sin(z) = \frac{e^{iz} - e^{-iz}}{2i}. \]
		Using these formulas,
		\begin{enumerate}
			\item[(a)] express $\cos(z)$ and $\sin(z)$ in rectangular coordinates $u(x,y) + iv (x,y)$ where $z = x + iy$.
			\item[(b)] show that the complex cosine and sine functions are analytic over $\mathbb{C}$ and calculate their derivatives.
		\end{enumerate}
		

\vspace{.5cm} % Space for work

\hrule

% Add more problems as needed...

\end{document}

%%% Local Variables:
%%% mode: latex
%%% TeX-master: t
%%% End:

%%% Local Variables:
%%% mode: latex
%%% TeX-master: t
%%% End:
