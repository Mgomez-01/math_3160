% --------------------------------------------------------------------
% LaTeX Template for Math Homework
% --------------------------------------------------------------------

\documentclass{article}

% --- PACKAGE IMPORTS ---
% These packages add functionality for math symbols, formatting, etc.
\usepackage[margin=.7in]{geometry}       % For setting page margins
\usepackage{amsmath, amssymb, amsthm}   % American Mathematical Society packages for advanced math
\usepackage{graphicx}                   % For including images
\usepackage{fancyhdr}                   % For creating custom headers and footers
\usepackage[colorlinks=true, urlcolor=blue, linkcolor=blue]{hyperref} % For clickable links
\usepackage{cancel}
\usepackage{array}
\usepackage{amsfonts}
\usepackage{amsxtra}
\usepackage{epsfig}
\usepackage{wasysym}
\usepackage{relsize}
\usepackage{tikz}
\tikzset{every picture/.style={scale=1.2}}
\renewcommand{\normalsize}{\fontsize{12}{20}\selectfont}

% custom commands
\newcommand{\myauthor}{Miguel Gomez}
\newcommand{\canceling}[2]{\textcolor{red}{\cancelto{\textcolor{black}{#1}}{\textcolor{black}{#2}}}}
\newcommand{\todo}[1]{\textcolor{blue}{TODO:#1}}
\newcommand{\der}[2]{\frac{d#1}{d#2}}
\newcommand{\part}[2]{\frac{\partial#1}{\partial#2}}

% --- DOCUMENT & AUTHOR INFORMATION ---
\title{Homework \# 4}
\author{
	MATH 3160 -- Complex Variables\\
	\myauthor
}
\date{Completed: \today}

% --- HEADER & FOOTER CONFIGURATION ---
% This section sets up the header that will appear on each page.
\pagestyle{fancy}
\fancyhf{} % Clears the default header and footer
\lhead{Math 3160 -- HW \# 4} % Left side of header
\rhead{\myauthor} % Puts the author's name on the right side
\rfoot{Page \thepage} % Puts the page number on the bottom right

\begin{document}

\maketitle % This command generates the title based on the information above.

% ====================================================================
% --- START OF PROBLEMS ---
% ====================================================================

\section*{Problem 1}
Find $f'(z)$ using differentiation rules.
\begin{enumerate}
	\item[(a)] $f(z) = 3z^2-2z+4$
	\item[(b)] $f(z) = (1-4z)^3$
	\item[(c)] $f(z) = \frac{z-1}{2z+1}$, assume $z\neq -1/2$
	\item[(d)] $f(z) = \frac{(z^2+1)^4}{z^2}$, assume $z\neq 0$
	\item[(e)] $f(z) = z \, e^{z^2 + 3}$.
\end{enumerate}

\subsection*{(a)}
\begin{align*}
	f(z)  & = 3z^2-2z+4 \\
	f'(z) & = 6z - 2
\end{align*}
\subsection*{(b)}
\begin{align*}
	f(z)  & = a(z) = (1-4z)^3 = b^3 \ |\ b = 1-4z                \\
	f'(z) & = \der{a}{b}\der{b}{z} = 3(1-4z)^2(-4) = -12(1-4z)^2
\end{align*}
\subsection*{(c)}
Assuming $z\neq -1/2$
\begin{align*}
	f(z)                          & = \frac{z-1}{2z+1} = \frac{f}{g}                        \\
	\der{f}{z}                    & = 1 \quad \quad \der{g}{z} = 2                          \\
	f'(z) = \frac{f'g - fg'}{g^2} & =  \frac{1(2z+1) - (z - 1)2}{(2z+1)^2}                  \\
	                              & = \frac{(2z-2z) + (1+2)}{(2z+1)^2} = \frac{3}{(2z+1)^2}
\end{align*}
\subsection*{(d)}
Assuming $z\neq 0$
\begin{align*}
	f(z)       & = \frac{(z^2+1)^4}{z^2} = \frac{a}{b}         \\
  \der{a}{z} & = 4(z^2+1)^3(2z)  \quad \quad \der{b}{z} = 2z \\
  f'(z) &= \frac{4(z^2+1)^3(2z)(z^2) - (z^2+1)^4(2z)}{z^4} \\
                   &= \frac{(2z)(z^2+1)^3[4(z^2) - (z^2+1)]}{z^4} \\
  &=\frac{2(z^2+1)^3[3z^2-1]}{z^3}
\end{align*}
\subsection*{(e)}
\begin{align*}
  f(z) &= z e^{z^2 + 3}\\
  f'(z) &= (1)e^{z^2 + 3} + (z)e^{z^2 + 3}(2z) = e^{z^2 + 3}(2z^2 + 1) 
\end{align*}

\hrule % Adds a horizontal line to separate problems.

\newpage
\section*{Problem 2}
Show that $f'(z_0)$ does not exist at any point $z_0$ in two ways: using the limit definition and using the Cauchy-Riemann equations. Here, $z = x + iy$ and $x,y \in \mathbb{R}$.
\begin{enumerate}
	\item[(a)] $f(z) = 2x+ixy^2$
	\item[(b)] $f(z) = e^{x}e^{-iy}$
\end{enumerate}
Via Cauchy-Riemann equations:
The Cauchy-Riemann equations are the following:
\begin{align*}
	\der{u}{x} & = \der{v}{y}  \\
	\der{u}{y} & = -\der{v}{x}
\end{align*}
\subsection*{(a)}
\begin{align*}
  f(z) &= 2x+ixy^2 \quad u(x,y) = 2x \quad v(x,y) =xy^2\\
  \der{u}{x} &= 2 \quad \der{u}{y}= 0 \quad  \der{v}{x} = y^2 \quad \der{v}{y} = 2xy\\
  \der{u}{x} &\neq \der{v}{y}\\
  &\therefore \text{not differentiable}
\end{align*}
\subsection*{(b)}
\begin{align*}
  f(z) &= e^{x}e^{-iy} = e^{x}(\cos{(-y)} + i\sin{(-y)}) \\
  u(x,y) &= e^{x}\cos{(-y)} \quad v(x,y) =e^{x}\sin{(-y)}\\
  \der{u}{x} &= e^{x}\cos{(y)} \quad \der{u}{y}= -e^{x}\sin{(y)} \quad  \der{v}{x} = -e^{x}\sin{(y)} \quad \der{v}{y} = -e^{x}\cos{(y)}\\
  e^{x}\cos{(y)} &= -e^{x}\cos{(y)} \to 2e^{x}\cos{(y)} = 0\\
  &\text{Only possible if $\cos{(y)}$ = 0} \\
  -e^{x}\sin{(y)} & = e^{x}\sin{(y)}  \to  2e^{x}\sin{(y)} = 0\\
       &\text{only possible if $\sin{(y)}$ is 0} \\
  &\text{Both $\sin$ and $\cos$ cannot be 0 simultaneously} \\
       &\therefore \text{not differentiable}
\end{align*}

\hrule
\vspace{.5cm} % Space between problems
\newpage
Now by using the limit definition:

Limit definition:
\begin{align*}
  \lim\limits_{z \to z_0}&\frac{f(z)-f(z_0)}{z-z_o}\\
  &= \lim\limits_{\Delta z \to 0} \frac{\Delta w}{\Delta z}\\
  \Delta w &= f(z_0 + \Delta z) - f(z_0) \\
  \Delta z &= z - z_0\\
\end{align*}
\subsection*{(a)}
\begin{align*}
  \Delta w &=  2(x_0+\Delta x)+i(x_0+\Delta x)(y_0+\Delta y)^2 - (2x_0+ix_0y_0^2)\\
           &= 2x_0+2\Delta x - 2x_0 + i(x_0+\Delta x)(y_0+\Delta y)^2 - ix_0y_0^2 \\
           &= 2\Delta x+ i(x_0+\Delta x)(y_0^2+2y_0\Delta y + \Delta y ^2)- ix_0y_0^2\\
           &= 2\Delta x+ i((x_0y_0^2+2x_0y_0\Delta y + x_0\Delta y ^2)+(\Delta xy_0^2+2\Delta xy_0\Delta y + \Delta x\Delta y ^2))- ix_0y_0^2\\
           &= 2\Delta x+ i((\canceling{0}{(x_0y_0^2- x_0y_0^2)}+2x_0y_0\Delta y + x_0\Delta y ^2)+(\Delta xy_0^2+2\Delta xy_0\Delta y + \Delta x\Delta y ^2))\\  
           &= 2\Delta x+ i((2x_0y_0\Delta y + x_0\canceling{0}{\Delta y ^2})+(\Delta xy_0^2+\canceling{0}{2\Delta xy_0\Delta y} + \Delta x\canceling{0}{\Delta y ^2}))\\
           &= 2\Delta x+ i(2x_0y_0\Delta y +\Delta xy_0^2)\\
  \frac{\Delta w}{\Delta z} &= \frac{2\Delta x+ i(2x_0y_0\Delta y +\Delta xy_0^2)}{\Delta x + i\Delta y}
\end{align*}
Approach with $\Delta y = 0$:
\begin{align*}
  \frac{\Delta w}{\Delta z} &= \frac{2\Delta x+ i(2x_0y_0(0) +\Delta xy_0^2)}{\Delta x + i(0)}\\
  &= \frac{2\Delta x+ i(\Delta xy_0^2)}{\Delta x} = 2+iy_0^2
\end{align*}
Approach with $\Delta x = 0$:
\begin{align*}
  \frac{\Delta w}{\Delta z} &= \frac{2(0)+ i(2x_0y_0\Delta y +(0)y_0^2)}{(0) + i\Delta y} \\
  &= \frac{i2x_0y_0\Delta y}{i\Delta y} = 2x_0y_0
\end{align*}
Different paths give different results. only true if $2x_0y_0 = 2+iy_0$ and these cannot be true since $x_0 $ and $y_0 \in \mathbb{R}$.
\newpage
\subsection*{(b)}
\begin{align*}
  f(z) &= e^{x}e^{-iy}\\
  \Delta w &= f(z_0 + \Delta z) - f(z_0) \\
  \Delta z &= z - z_0\\
  \Delta w &=e^{(x_0+\Delta x)}e^{-i(y_0+\Delta y)} - e^{x_0}e^{-iy_0} \\
       &=e^{x_0}e^{\Delta x}e^{-iy_0}e^{-i\Delta y} - e^{x_0}e^{-iy_0} \\
       &=e^{x_0}e^{-iy_0}e^{\Delta x}e^{-i\Delta y} - e^{x_0}e^{-iy_0} \\
       &=e^{x_0}e^{-iy_0}(e^{\Delta x}e^{-i\Delta y} - 1) \\
       &=e^{x_0}e^{-iy_0}(e^{\Delta x-i\Delta y} - 1) \\
 \frac{\Delta w}{\Delta z} &=\frac{e^{x_0}e^{-iy_0}(e^{\Delta x-i\Delta y} - 1)}{\Delta x + i\Delta y}
\end{align*}
Approach with $\Delta y = 0$:
\begin{align*}
  \frac{\Delta w}{\Delta z}&= \frac{e^{x_0}e^{-iy_0}(e^{\Delta x-i(0)} - 1)}{\Delta x + i(0)} \\
                           &= \frac{e^{x_0}e^{-iy_0}(e^{\Delta x} - 1)}{\Delta x} \\
                           &= e^{x_0}e^{-iy_0}\lim\limits_{\Delta x \to 0}\frac{(e^{\Delta x} - 1)}{\Delta x} \\
                           &\text{limit definition of exponential resolves to 1}\\
                           &= = e^{x_0}e^{-iy_0}
\end{align*}
Approach with $\Delta x = 0$:
\begin{align*}
  \frac{\Delta w}{\Delta z}&= \frac{e^{x_0}e^{-iy_0}(e^{(0)-i\Delta y} - 1)}{(0) + i\Delta y} \\
                           &= \frac{e^{x_0}e^{-iy_0}(e^{-i\Delta y} - 1)}{i\Delta y} \\
                           &\text{limit definition of exponential resolves to -1}\\
                           &= = -e^{x_0}e^{-iy_0}
\end{align*}
Different paths give different results:
\begin{align*}
  \therefore \text{Limit DNE}
\end{align*}
\vspace{.5cm} % Space between problems
\hrule

\newpage
\section*{Problem 3}
Using the exponential function $e^{z}$, we can now define the complex cosine and sine function for any $z \in \mathbb{C}$ as follows:
\[  \cos(z) = \frac{e^{iz} + e^{-iz}}{2} \]

and
\[  \sin(z) = \frac{e^{iz} - e^{-iz}}{2i}. \]
Using these formulas,
\begin{enumerate}
	\item[(a)] express $\cos(z)$ and $\sin(z)$ in rectangular coordinates $u(x,y) + iv (x,y)$ where $z = x + iy$.
	\item[(b)] show that the complex cosine and sine functions are analytic over $\mathbb{C}$ and calculate their derivatives.
\end{enumerate}

\subsection*{(a)}
\begin{align*}
  \cos{(z)} &= \frac{e^{iz} + e^{-iz}}{2}  = \frac{e^{i(x+iy)} + e^{-i(x+iy)}}{2} \\
          &= \frac{e^{ix-y} + e^{-ix+y}}{2} = \frac{e^{ix}e^{-y} + e^{-ix}e^{y}}{2} \\
          &= e^{ix} = (\cos{(x)} + i\sin{(x)}) \\
          &= e^{-ix} = (\cos{(x)} - i\sin{(x)}) \\
          &= \frac{e^{ix}e^{-y} + e^{-ix}e^{y}}{2} = \frac{(\cos{(x)} + i\sin{(x)})e^{-y} + (\cos{(x)} - i\sin{(x)})e^{y}}{2} \\
  &= \frac{\cos{(x)}(e^{-y} + e^{y}) + i\sin{(x)}(e^{-y}- e^{y})}{2} \\
  &= \frac{\cos{(x)}(e^{-y} + e^{y})}{2} + i\frac{\sin{(x)}(e^{-y}- e^{y})}{2} \\
\end{align*}
\subsection*{(b)}
\begin{align*}
  \sin{(z)} &= \frac{e^{iz} - e^{-iz}}{2i}  = \frac{e^{i(x+iy)} - e^{-i(x+iy)}}{2i} \\
          &= \frac{e^{ix-y} - e^{-ix+y}}{2i} = \frac{e^{ix}e^{-y} - e^{-ix}e^{y}}{2i} \\
          &= e^{ix} = (\cos{(x)} + i\sin{(x)}) \\
          &= e^{-ix} = (\cos{(x)} - i\sin{(x)}) \\
          &= \frac{e^{ix}e^{-y} - e^{-ix}e^{y}}{2i} = \frac{(\cos{(x)} + i\sin{(x)})e^{-y} - (\cos{(x)} - i\sin{(x)})e^{y}}{2i} \\
  &= \frac{\cos{(x)}(e^{-y} - e^{y}) + i\sin{(x)}(e^{-y}+ e^{y})}{2i} \\
            &= \frac{\cos{(x)}(e^{-y} - e^{y})}{2i} + \frac{\sin{(x)}(e^{-y}+ e^{y})}{2} \\
  &=  \frac{\sin{(x)}(e^{-y}+ e^{y})}{2} -i\frac{\cos{(x)}(e^{-y} - e^{y})}{2} \\
\end{align*}
These two in rectangular coordinates appear to be the same thing with their respective $\sin$ and $\cos$ terms swapped. Another place where we see the $\sin$ and $\cos$ terms swap is when we perform a rotation by $90^\circ$. I suspect that including $i$ along with $z$ is performing a rotation on $z$ by $90^\circ$ and that is shown in the work. Therefore, proving the case for $\cos$ also proves the case for $\sin$ because we know a. priori that $\sin$ and $\cos$ are derivatives of each other in a cycle.

\begin{align*}
  \der{f}{z} &= \text{product rule here after splitting}\\
  & \frac{\cos{(x)}(e^{-y} + e^{y})}{2}\\
  & i\frac{\sin{(x)}(e^{-y}- e^{y})}{2}
\end{align*}
\vspace{.5cm} % Space for work

\hrule

% Add more problems as needed...

\end{document}

%%% Local Variables:
%%% mode: latex
%%% TeX-master: t
%%% End:

%%% Local Variables:
%%% mode: latex
%%% TeX-master: t
%%% End:
