% --------------------------------------------------------------------
% LaTeX Template for Math Homework
% --------------------------------------------------------------------

\documentclass{article}

% --- PACKAGE IMPORTS ---
% These packages add functionality for math symbols, formatting, etc.
\usepackage[margin=.7in]{geometry}       % For setting page margins
\usepackage{amsmath, amssymb, amsthm}   % American Mathematical Society packages for advanced math
\usepackage{graphicx}                   % For including images
\usepackage{fancyhdr}                   % For creating custom headers and footers
\usepackage[colorlinks=true, urlcolor=blue, linkcolor=blue]{hyperref} % For clickable links
\usepackage{cancel}
\usepackage{array}
\usepackage{amsfonts}
\usepackage{amsxtra}
\usepackage{epsfig}
\usepackage{wasysym}
\usepackage{relsize}
\usepackage{tikz}
\tikzset{every picture/.style={scale=1.2}}
\renewcommand{\normalsize}{\fontsize{12}{20}\selectfont}

% custom commands
\newcommand{\myauthor}{Miguel Gomez}
\newcommand{\canceling}[2]{\textcolor{red}{\cancelto{\textcolor{black}{#1}}{\textcolor{black}{#2}}}}
\newcommand{\todo}[1]{\textcolor{blue}{TODO:#1}}
% Save the original commands
\let\oldcos\cos
\let\oldsin\sin
\let\oldcosh\cosh
\let\oldsinh\sinh

% Redefine with automatic parentheses
\renewcommand{\cos}[1]{\oldcos\left(#1\right)}
\renewcommand{\sin}[1]{\oldsin\left(#1\right)}
\renewcommand{\cosh}[1]{\oldcosh\left(#1\right)}
\renewcommand{\sinh}[1]{\oldsinh\left(#1\right)}

\newcommand{\der}[2]{\frac{d#1}{d#2}}
\newcommand{\secder}[2]{\frac{d^2#1}{d#2^2}}
\newcommand{\parder}[2]{\frac{\partial#1}{\partial#2}}
\newcommand{\secparder}[2]{\frac{\partial^2#1}{\partial#2^2}}

% --- DOCUMENT & AUTHOR INFORMATION ---
\title{Homework \#5}
\author{
	MATH 3160 -- Complex Variables\\
	\myauthor
}
\date{Completed: \today}

% --- HEADER & FOOTER CONFIGURATION ---
% This section sets up the header that will appear on each page.
\pagestyle{fancy}
\fancyhf{} % Clears the default header and footer
\lhead{Math 3160 -- HW \# 5} % Left side of header
\rhead{\myauthor} % Puts the author's name on the right side
\rfoot{Page \thepage} % Puts the page number on the bottom right

\begin{document}

\maketitle % This command generates the title based on the information above.

% ====================================================================
% --- START OF PROBLEMS ---
% ====================================================================

\section*{Problem 1}
Consider the analytic function $f(z) = z e^{z^2}$.
\begin{enumerate}
	\item [(a)] Show that the function $u(x,y) = x \ e^{(x^2 - y^2)} \cos{2xy} - y \ e^{(x^2 - y^2)} \sin{2xy}$ is the real component of $f(z)$.
	\item [(b)] What is a harmonic conjugate for $u(x,y)$?
	\item [(c)] Without computing the second partial derivatives of $u(x,y)$, explain why you know that $u(x,y)$ is harmonic.
\end{enumerate}

\vspace{.5cm} % Space between problems

\hrule % Adds a horizontal line to separate problems.
Expanding $f(z) = z e^{z^2}$ to see what the $u$ and $v$ turn out to be.
\subsection*{(a)}
\begin{align*}
  f(z) &= z e^{z^2} = (x+iy)e^{(x+iy)^2} = (x+iy)e^{x^2-y^2+2ixy} = (x+iy)e^{x^2-y^2}e^{2ixy}\\
  &= e^{x^2-y^2}(x+iy)(\cos{2xy}+i\sin{2xy}) \\
       &= e^{x^2-y^2}(x\cos{2xy}+ix\sin{2xy} + iy\cos{2xy}+i^2y\sin{2xy})\\
       &=  e^{x^2-y^2}(x\cos{2xy}-y\sin{2xy} + i(x\sin{2xy} + y\cos{2xy}))\\
       &\therefore u(x,y) = e^{x^2-y^2}(x\cos{2xy}-y\sin{2xy})
\end{align*}
\subsection{(b)}
Utilizing the imaginary part of $f$, $v$ can serve as a conjugate up to an arbitrary constant, which we could set to 0 to recover $f$. 
\begin{align*}
v(x,y) &= e^{x^2-y^2} (x\sin{2xy} + y\cos{2xy}) + C
\end{align*}
\subsection{(c)}
Since we know that $f$ is analytic as stated, and $u$ is the real part of $f$, an analytic function, then $u$ must be harmonic. This is due to the fact that the real and imaginary parts of any analytic function are harmonic functions.
\vspace{.2cm}
\hrule

\section*{Problem 2}
Consider the function  $u(x,y) = x^3 - 3 x y^2 - 3 x^2 y + y^3$.

\begin{enumerate}
	\item [(a)] Show that $u(x,y)$ is harmonic.
	\item [(b)] Find a harmonic conjugate for $u(x,y)$.
\end{enumerate}


\vspace{.5cm} % Space between problems

\hrule
\subsection*{(a)}
For this, we can start by showing that the expression for $u$ satisfies the Laplacian:
\begin{align*}
  \secparder{u}{x} &+ \secparder{u}{y} = 0\\
  \parder{u}{x} &= 3x^2 - 3 y^2 - 6xy\\
  \secparder{u}{x} &= 6x - 6y\\
  \parder{u}{y} &= - 6 x y - 3 x^2  + 3y^2\\
  \secparder{u}{y} &= - 6 x  + 6y\\
  \secparder{u}{x} &+ \secparder{u}{y} = 6x - 6y + (- 6 x  + 6y) = 0\\
  \therefore &\ u(x,y)\ \text{is harmonic.}
\end{align*}
\subsection*{(b)}
For a conjugate, we can back solve to get the $v$ expression that works.
\begin{align*}
  \parder{u}{x} &= \parder{v}{y} \\
  v(x,y) = \int 3x^2 - 3 y^2 - 6xy dy &= 3x^2y - y^3 - 3xy^2 + G(x)\\
  \parder{u}{y} &= -\parder{v}{x} \\
  v(x,y) = \int  6 x y + 3 x^2  - 3y^2 dx &= 3x^2y + x^3 -3x^2y + G(y)\\
  \therefore  v(x,y)_1 &= v(x,y)_2 \\
  G(x) &= x^3 + C\\
  G(y) &= -y^3 + C\\
  \therefore v(x,y) &= \boxed{ 3x^2y + x^3 -3x^2y + -y^3 + C}
\end{align*}
\newpage
\section*{Problem 3}
Recall we learned of the following fact in class:

\begin{center}
	{\em Let $u(x,y)$ be a harmonic function defined on a simply connected domain $D$. \\ Then $u(x,y)$ has a harmonic conjugate on $D$.}
\end{center}

\begin{enumerate}
	\item [(a)] Show that $u(x,y) = \ln(\sqrt{x^2 + y^2})$ is a harmonic function.

	\item [(b)] What is the domain of definition of $u(x,y)$?

	\item [(c)] An aside: show that if $f(z)$ and $g(z)$ are two analytic functions on the same domain $D$, and we have $\text{Re}(f(z)) = \text{Re}(g(z))$ for all $z \in D$, then $f(z) = g(z) + c$ for some constant $c \in \mathbb{C}$.

          [{\em Hint: show that the function $h(z) = f(z) - g(z)$ has $\text{Re}(h(z)) = 0$, and then use a result from class to conclude $h(z)$ is a constant.}]

	\item [(d)] Explain why $u(x,y)$ does {\it not} have a harmonic conjugate on its domain.

          [{\em Hint: if such a conjugate existed, then $u(x,y)$ would be the real component of some analytic function $f(z)$, but $u(x,y)$ is already the real component of a familiar analytic function, which is discontinuous at its branch cut}]

	\item Why does this not contradict the fact from class?
\end{enumerate}

\vspace{.5cm} % Space for work

\hrule
\subsection*{(a)}
\begin{align*}
  u(T) = \ln(T),\ \ T(S) &= \sqrt{S}, \ \ S = x^2+y^2 \\ 
  \der{u}{T} = \frac{1}{T} \ \ \der{T}{S} &= \frac{1}{2\sqrt{S}}\ \ \parder{S}{x|y} = 2x|2y\\
  \parder{u}{x} = \parder{}{x}(\ln{((x^2+y^2)^{\frac{1}{2}})}) &= ((x^2+y^2)^{-\frac{1}{2}})\left(\frac{1}{2}(x^2+y^2)^{-\frac{1}{2}}\right)(2x) \\
  &=\left(\frac{2x}{2(x^2+y^2)}\right)\\
  \parder{u}{x} &=\left(\frac{x}{x^2+y^2}\right)\\
  \parder{u}{y} &= \left(\frac{y}{x^2+y^2}\right)
\end{align*}
\begin{align*}
  \secparder{u}{x} =\frac{f'g-fg'}{g^2} &= \left(\frac{(x)'(x^2+y^2)-(x)(x^2+y^2)'}{(x^2+y^2)^2}\right)\\
                                        &= \left(\frac{(x^2+y^2)-(x)(2x)}{(x^2+y^2)^2}\right)\\
  \secparder{u}{x} &= \left(\frac{(y^2-x^2)}{(x^2+y^2)^2}\right)\\
  \secparder{u}{y}&= \left(\frac{(x^2-y^2)}{(x^2+y^2)^2}\right)\\
  \secparder{u}{x} &+ \secparder{u}{y} =  \frac{y^2-x^2 + x^2-y^2}{(x^2+y^2)^2} = 0
\end{align*}
\subsection*{(b)}
\begin{align*}
  \ln{(\sqrt{x^2+y^2})} &= \frac{1}{2}\ln{(x^2+y^2)} = \frac{1}{2}\ln{(|z|)}
\end{align*}
Domain of definition is anywhere that the magnitude is not zero.
\begin{align*}
  \mathbb{C}\setminus \{0\}
\end{align*}
\subsection*{(c)}
Defining a function $h=f-g$ as the hint suggests. If both $f$ and $g$ are analytic and their real parts are the same:
\begin{align*}
  h(z) &= f(z) - g(z) = g(z) + c - g(z) =  i*(\text{Im}{(f)} + \text{Im}(g)) 
\end{align*}
Since $h$ is equal to $c$ from this definition, because both $f$ and $g$ are analytic, then we know that the real part of $h$ is $0$, meaning we have $u(x,y) = 0$ for  $h$. From the CR equations, we must have the partial derivative of $h_{u_x}$ be the same as $h_{u_y}$. but we know that this should be $0$, therefore all of $h$ must be constant.
\subsection*{(d)}
It does not because it is a function that has a behavior that exhibits periodicity due to the arg$(z)$. Because of this, the answers repeat for integer values $k$. The inclusion of the branch cut means that it cannot be continuous across branch cuts.
\subsection*{Why no contradiction?}
If I recall correctly, it is because we use the branch cut to define the domain and this kind of domain is not simply connected. Because of this, it does not contradict the definitions we discussed.
\newpage
\section*{Problem 4}
Find the following values, on the branches given:
\begin{enumerate}
	\item [(a)] $ \log(3) \ \ (-2\pi \leq \theta < 0)$ 
	\item [(b)] $ \log(-1 + i) \ \ (-\pi/2 < \theta \leq 3 \pi / 2)$
	\item [(c)] $ \log(1 - i \sqrt 3) \ \ (\pi \leq \theta < 3 \pi)$.
\end{enumerate}
\hrule
\vspace{.5cm} % Space for work

\begin{align*}
  \log(z) &= \log{|z|} + iarg(z) = \log{|z|} + i(Arg(z) + 2\pi k)\ \forall k \in \mathbb{Z}
\end{align*}
\subsection*{(a)}
$ \log{(3)} \ \ (-2\pi \leq \theta < 0)$
\begin{align*}
  \log{(3)} &=  \log{|3|} + i(Arg(3) + 2\pi k)= \log{(3)} + i(0 +  2\pi k)\\
  \log{(3)} &+ i(2\pi k)\ k \in [-1,0) \quad \k = -1\\
  &= \log{(3)} - i2\pi
\end{align*}
\subsection*{(b)}
$ \log(-1 + i) \ \ (-\pi/2 < \theta \leq 3 \pi / 2)$
\begin{align*}
  \log{(-1 + i)} &=  \log{|-1 + i|} + i(Arg(-1 + i) + 2\pi k)= \log{(\sqrt{2})} + i\left(\frac{3\pi}{4} +  2\pi k\right)\\
  \log{(\sqrt{2})} &+ i\left(\frac{3\pi}{4}+2\pi k\right)\ k \in \left(-\frac{1}{4}, \frac{3}{4}\right]\\
                 &= \log{(\sqrt{2})} + i\left(\frac{3\pi}{4}+2\pi k\right) \ k = 0\\
                 &= \log{(\sqrt{2})} + i\frac{3\pi}{4}
\end{align*}
\subsection*{(c)}
$ \log(1 - i \sqrt 3) \ \ (\pi \leq \theta < 3 \pi)$
\begin{align*}
  \log{(1-i\sqrt{3})} &=  \log{|1-i\sqrt{3}|} + i(Arg(1-i\sqrt{3}) + 2\pi k)= \log{(2)} - i\left(\frac{\pi}{3} +  2\pi k\right)\\
  &= \log{(2)} - i\left(\frac{\pi}{3} +  2\pi k\right) \ k\in \left[\frac{1}{2},\frac{3}{2}\right) \quad k = 1\\
  &= \log{(2)} +i\frac{5\pi}{3} 
\end{align*}
\newpage
\section*{Problem 5}
Recall that power functions are defined by $z^c = e^{c \log(z)}$. In this exercise, we compute all power functions by using the branch $(0 \leq \theta < 2 \pi)$ for $\log(z)$.
\begin{enumerate}
\item [(a)] For $z = -i$ and $c = i$, compute the values of $(z^c)^2$, $(z^2)^c$, and $z^{(2c)}$.
\item [(b)] With the notation as in (a), which of these are true or false?
  $$ (z^c)^2 = (z^2)^c, \hspace{20pt} (z^c)^2 = z^{(2c)}, \hspace{20pt} (z^2)^c = z^{(2c)}.$$
\end{enumerate}
\begin{align*}
  \text{case: } (z^c)^2 &= (z^2)^c \\
  ((-i)^i)^2 &= (e^{i(\log{(-i)})})^2\\
  e^{i(\log{(-i)})} &= e^{i(\log{(|-i|) + \text{arg}(-i)+2\pi k})}\\
          &= e^{i(0 + \frac{3\pi}{2}+2\pi k)}\\
          &= e^{i(\frac{3\pi}{2}+0)} = e^{i(\frac{3\pi}{2})}\\
  (e^{i(\frac{3\pi}{2})})^2 &= e^{i3\pi}\\
  (z^2)^c &= ((-i)^2)^i = ((-1)^2(i)^2)^i \\
          &= (-1)^i = e^{i(\log{(-1)})}= e^{i(\Log{|-1|} + \text{arg}(-1) + 2\pi k)}\\
          &= e^{i(\Log{1} + \pi + 0)} = e^{i\pi} \\
  \therefore (z^c)^2 &\neq (z^2)^c \text{ given the branch cut.}
\end{align*}
\begin{align*}
  \text{case: } (z^c)^2 &= z^{(2c)} \\
  (z^c)^2 &= e^{i3\pi}\\
  z^{(2c)} &= (-i)^{(2i)} = e^{2i(\log{(-i)})}\\
  &= e^{2i(\log{|-i| + \text{arg}(-i) + 2\pi k})}\\
  &= e^{2i(0 + \frac{3\pi}{2} + 0)} = e^{i3\pi}\\
  \therefore (z^c)^2 &= z^{(2c)} \text{ given the branch cut.}
\end{align*}
\begin{align*}
  \text{case: } (z^2)^c &= z^{(2c)} \\
  (z^2)^c &= e^{i\pi}\\
  z^{(2c)} &= e^{i3\pi}\\
\therefore (z^2)^c &\neq z^{(2c)} \text{ given the branch cut.}  
\end{align*}



\end{document}

%%% Local Variables:
%%% mode: latex
%%% TeX-master: t
%%% End:





