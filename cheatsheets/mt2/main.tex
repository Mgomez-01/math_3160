\documentclass{../cheatsheet}
\title{Complex Variables CheatSheet II}
\author{Miguel * adapted for Complex Analysis}
\usepackage{amssymb}
\usepackage{amsmath} % For align environment and better math formatting
\usepackage{multicol}
\usepackage{txfonts}
\usepackage{graphicx}
\usepackage{pgfplots}
\pgfplotsset{compat=1.5}
\usepackage{hyperref}

\begin{document}

\begin{multicols}{3}
\section{Harmonic Functions}

\textbf{Definition}
A real-valued function $u(x,y)$ is \textbf{harmonic} in a domain $D$ if it has continuous second-order partials and satisfies \textbf{Laplace's equation}:
$$\nabla^2 u = \frac{\partial^2 u}{\partial x^2} + \frac{\partial^2 u}{\partial y^2} = 0$$
\begin{itemize}
    \item If $f(z) = u+iv$ is analytic, then $u$ and $v$ are both harmonic.
\end{itemize}

\textbf{Harmonic Conjugate}
$v(x,y)$ is a \textbf{harmonic conjugate} of $u(x,y)$ if $f(z) = u(x,y) + iv(x,y)$ is analytic. (This means they must satisfy the C-R equations).

\textbf{Methods for Finding $v$ from $u$}
\begin{enumerate}
    \item \textbf{Integrate C-R Equations}
    \begin{itemize}
        \item \textbf{Step 1:} Use $\frac{\partial v}{\partial y} = \frac{\partial u}{\partial x}$. Integrate w.r.t. $y$:
        $$v(x,y) = \int \frac{\partial u}{\partial x} dy + h(x)$$
        (where $h(x)$ is an unknown function of $x$).
        \item \textbf{Step 2:} Use $\frac{\partial v}{\partial x} = -\frac{\partial u}{\partial y}$. Differentiate the result from Step 1 w.r.t. $x$:
        $$\frac{\partial v}{\partial x} = \frac{\partial}{\partial x} \left( \int \frac{\partial u}{\partial x} dy \right) + h'(x)$$
        \item \textbf{Step 3:} Set equal and solve for $h'(x)$:
        $$h'(x) = -\frac{\partial u}{\partial y} - \frac{\partial}{\partial x} \left( \int \frac{\partial u}{\partial x} dy \right)$$
        \item \textbf{Step 4:} Integrate $h'(x)$ to find $h(x)$ and add the final constant $C$.
    \end{itemize}

    \item \textbf{Total Differential (Line Integral)}
    \begin{itemize}
        \item The total differential for $v$ is $dv = \frac{\partial v}{\partial x} dx + \frac{\partial v}{\partial y} dy$.
        \item \textbf{Step 1:} Use C-R to write $dv$ in terms of $u$:
        $$dv = \left(-\frac{\partial u}{\partial y}\right) dx + \left(\frac{\partial u}{\partial x}\right) dy$$
        \item \textbf{Step 2:} Integrate $dv$ from a fixed point $(x_0, y_0)$ to $(x,y)$ along a simple path (e.g., $(x_0, y_0) \to (x, y_0) \to (x, y)$).
        $$v(x,y) = \int_{x_0}^x -\frac{\partial u}{\partial y}(t, y_0) dt + \int_{y_0}^y \frac{\partial u}{\partial x}(x, t) dt + C$$
    \end{itemize}

    \item \textbf{Inspection / Guess $f(z)$}
    \begin{itemize}
        \item Works for simple polynomials.
        \item Let $y=0$, giving $u(x,0)$.
        \item Try to guess the analytic function $f(z)$ by replacing $x$ with $z$.
        \item \textbf{Ex:} $u(x,y) = x^2-y^2$. Let $y=0 \implies u(x,0) = x^2$.
        \item \textbf{Guess:} $f(z) = z^2$.
        \item \textbf{Check:} $f(z) = (x+iy)^2 = (x^2-y^2) + i(2xy)$.
        \item This works! $u=x^2-y^2$ and $v=2xy$.
    \end{itemize}
\end{enumerate}

\section{Contour Integration}

\textbf{Parametrization}
A contour $C$ is a curve $z(t) = x(t) + iy(t)$ for $a \le t \le b$.
\begin{itemize}
    \item $z'(t) = \frac{dz}{dt} = x'(t) + iy'(t)$.
    \item $C$ is \textbf{simple} if it doesn't cross itself.
    \item $C$ is \textbf{closed} if $z(a) = z(b)$.
\end{itemize}

\textbf{Definition of the Contour Integral}
$$\int_C f(z) dz = \int_a^b f(z(t)) z'(t) dt$$

\textbf{Common Parametrizations}
\begin{itemize}
    \item \textbf{Line Segment} from $z_1$ to $z_2$:
    \begin{itemize}
        \item $z(t) = z_1 + t(z_2 - z_1)$, for $0 \le t \le 1$.
        \item $z'(t) = z_2 - z_1$.
    \end{itemize}
    \item \textbf{Circle} $|z-z_0|=R$:
    \begin{itemize}
        \item \textbf{CCW (Positive):} $z(t) = z_0 + Re^{it}$
        \item $z'(t) = iRe^{it}$, for $0 \le t \le 2\pi$.
        \item \textbf{CW (Negative):} $z(t) = z_0 + Re^{-it}$
        \item $z'(t) = -iRe^{-it}$, for $0 \le t \le 2\pi$.
    \end{itemize}
\end{itemize}

\textbf{Properties of Integrals}
\begin{itemize}
    \item \textbf{Linearity:} $\int_C (\alpha f + \beta g) dz = \alpha \int_C f dz + \beta \int_C g dz$.
    \item \textbf{Path Reversal:} $\int_{-C} f(z) dz = - \int_C f(z) dz$.
    \item \textbf{Additivity:} $\int_{C_1+C_2} f(z) dz = \int_{C_1} f(z) dz + \int_{C_2} f(z) dz$.
\end{itemize}

\textbf{ML-Inequality (Estimation Bound)}
If $|f(z)| \le M$ for all $z$ on $C$, and $L$ is the arc length of $C$, then:
$$ \left| \int_C f(z) dz \right| \le M \cdot L $$

\section{Antiderivatives & FTC}

\textbf{Definition: Antiderivative}
A function $F(z)$ is an \textbf{antiderivative} of $f(z)$ in a domain $D$ if $F'(z) = f(z)$ for all $z \in D$.

\textbf{Complex Fundamental Theorem of Calculus}
If $f(z)$ is continuous in a domain $D$ and has an antiderivative $F(z)$ in $D$, then for any contour $C$ in $D$ from $z_1$ to $z_2$:
$$ \int_C f(z) dz = F(z_2) - F(z_1) $$
\textbf{Key Consequences:}
\begin{itemize}
    \item \textbf{Path Independence:} The integral's value depends only on the endpoints.
    \item \textbf{Closed Loop Theorem:} If $C$ is closed ($z_1=z_2$), then $\oint_C f(z) dz = 0$.
\end{itemize}

\textbf{Existence of an Antiderivative}
An analytic function $f(z)$ has an antiderivative in a domain $D$ \textbf{if and only if} $D$ is \textbf{simply connected}.
\begin{itemize}
    \item \textbf{Ex:} $f(z)=1/z$ is analytic on $D = \mathbb{C}\setminus\{0\}$, which is \textbf{not} simply connected. $\oint_{|z|=1} \frac{1}{z} dz = 2\pi i \ne 0$. Thus $1/z$ has no single antiderivative (like $\text{Log}(z)$) on this domain.
\end{itemize}


\section{Cauchy Theorems (Deformation)}

\textbf{Cauchy-Goursat Theorem}
If $f(z)$ is analytic at all points \textbf{inside and on} a simple closed contour $C$, then:
$$ \oint_C f(z) dz = 0 $$
(Goursat's contribution was proving this without assuming $f'(z)$ is continuous).

\textbf{Principle of Deformation of Paths}
If $C_1$ and $C_2$ are two simple closed, positively oriented (CCW) contours, and $f(z)$ is analytic on both contours and in the region \textbf{between} them, then:
$$ \oint_{C_1} f(z) dz = \oint_{C_2} f(z) dz $$
This allows "deforming" a contour around singularities.

\textbf{Multiply Connected Domains}
Let $C$ be an outer CCW contour and $C_1, \dots, C_n$ be inner contours, all \textbf{clockwise (CW)}. If $f(z)$ is analytic in the region, then:
$$ \oint_C f(z) dz + \sum_{k=1}^n \oint_{C_k} f(z) dz = 0 $$
If all contours are \textbf{CCW}:
$$ \oint_C f(z) dz = \sum_{k=1}^n \oint_{C_k} f(z) dz $$
(The integral around the outer boundary equals the sum of integrals around the "holes").

\textbf{Keyhole Contours}
Used to make a multiply connected domain simply connected by using a "cut". This is common for branch cuts. The integral along the cut path $L_1$ and the return path $L_2$ cancel each other out.

\section{Cauchy's Integral Formulas}
(Review: Using the formulas to compute integrals)
(Assumes $f(z)$ is analytic inside and on a simple closed CCW contour $C$)

\textbf{The Integral Formula}
Gives the value of $f(z)$ at any point $z_0$ \textbf{inside} $C$.
$$ f(z_0) = \frac{1}{2\pi i} \oint_C \frac{f(z)}{z-z_0} dz $$
\textbf{To compute $\oint_C \frac{g(z)}{z-z_0} dz$:}
\begin{enumerate}
    \item Identify $z_0$. If $z_0$ is \textbf{outside} $C$, the integral is 0 (by Cauchy-Goursat, if $g(z)$ is analytic).
    \item If $z_0$ is \textbf{inside} $C$, let $f(z) = g(z)$.
    \item The integral is $\oint_C \frac{f(z)}{z-z_0} dz = 2\pi i \cdot f(z_0)$.
\end{enumerate}

\textbf{The Generalized Formula (for derivatives)}
Gives the $n$-th derivative of $f(z)$ at $z_0$.
$$ f^{(n)}(z_0) = \frac{n!}{2\pi i} \oint_C \frac{f(z)}{(z-z_0)^{n+1}} dz $$
\textbf{To compute $\oint_C \frac{g(z)}{(z-z_0)^{n+1}} dz$:}
\begin{enumerate}
    \item Identify $z_0$ (must be inside $C$) and the power $n+1$. This gives you $n$.
    \item Let $f(z) = g(z)$ (the analytic numerator).
    \item Find the $n$-th derivative, $f^{(n)}(z)$.
    \item Evaluate $f^{(n)}(z_0)$.
    \item The integral is $\oint_C \frac{f(z)}{(z-z_0)^{n+1}} dz = \frac{2\pi i}{n!} \cdot f^{(n)}(z_0)$.
\end{enumerate}
\textbf{Example:} $\oint_{|z|=2} \frac{e^{z}}{(z-1)^3} dz$
\begin{itemize}
    \item $z_0 = 1$ (inside $C$).
    \item $n+1 = 3 \implies n=2$.
    \item $f(z) = e^z$.
    \item $f'(z) = e^z$, $f''(z) = e^z$.
    \item $f''(1) = e^1 = e$.
    \item Result: $\frac{2\pi i}{2!} f''(1) = \frac{2\pi i}{2} (e) = \pi i e$.
\end{itemize}

\end{multicols}
\end{document}